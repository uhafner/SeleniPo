\chapter{Fazit}
\label{sec:fazit}
Das manuelle Testen von Software ist ein Bereich der Softwerentwicklung der in der Literatur mittlerweile gut und umfassend beleuchtet ist. Anders verhält es sich mit der Testautomatisierung. Dieser Bereich des Testens ist zwar schon lange bekannt wird in der Literatur jedoch immer noch nicht umfassend genug behandelt. Mit dieser Arbeit wurde ein Werk geschaffen welches sich mit einer Sparte der Softwareentwicklung befasst die noch Entwicklungspotential in sich birgt. 
Zunächst wurden die Grundlagen des Testens im allgemeinen vorgestellt und herausgearbeitet, dass Testen eine sinnvolle Möglichkeit ist um die Qualität von Softwareprodukten zu verbessern. 
Die Automatisierung des Testens wurde im laufe der Arbeit näher beleuchtet. Dazu wurden die verschiedenen Bereiche und Möglichkeiten der Testautomatisierung aufgezeigt so wie die Vor- und Nachteile herausgearbeitet.
Die Testautomatisierung wurde als Möglichkeit identifiziert, die Erfolge die mit Hilfe von Softwaretests erzielt werden können weiter zu verbessern und die dabei eingesetzten Mittel zu reduzieren. Als ein Problem der Testautomatisierung wurde der oft erhöhte initiale Mehraufwand bei der Erstellung von Testfällen genannt und als Möglichkeit für eine Verbesserung in diesem Bereich erkannt.
Als Beispiel für ein Testautomatisierungs-Tool welches als Schnittstelle die Oberfläche einer Webanwendung verwendet wurde Selenium vorgestellt.
Auch für dieses Tool hat sich gezeigt, dass der initiale Mehraufwand bei der Testfallerstellung verglichen mit der manuellen Durchführung der Testfälle erhöht ist. Darüber hinaus ergab sich, dass dieser Aufwand noch weiter steigt wenn Selenium wie vorgeschlagen in Verbindung mit dem Page Object Pattern verwendet wird.
Für den Einsatz von Selenium in Verbindung mit dem Page Object Pattern wurde daher eine Softwarelösung entwickelt und vorgestellt welche den Aufwand bei der Erstellung von automatischen Tests reduziert.
Die Softwarelösung in Form eines Page Object Generators vereinfacht dazu die Erstellung der Page Objects die beim Einsatz des Page Object Patterns benötigt werden. Page Objects müssen nicht mehr händisch programmiert werden sondern können teilautomatisiert aus dem Quelltext einer Webseite generiert werden. Der Page Object Generator trägt damit dazu bei, den initialen Mehraufwand bei der Erstellung des Testfälle zu reduzieren.

\section{Ausblick}
\label{ausblick}
Der Page Object Generator hat einen Stand erreicht der sich für den Produktiven Einsatz eignet. Praxistests haben jedoch gezeigt, dass Generator so wie der zugehörige Testharness durchaus noch Verbesserungspotenzial bieten.
Ein Maximum an Zeitersparnis lässt sich mit dem Page Object Generator erreichen, wenn der Konfigurationsaufwand von Testprojekt und Generator möglichst gering gehalten wird.
Ein sinnvoller Ansatzpunkt um das Projekt SeleniPo weiter voranzutreiben wäre es daher die Entwicklung des Testharness zu Verbessern. Bei der Implementierung des Testharness wie er in Kapitel \ref{sec:selenipotestharness} beschrieben ist, handelt es sich um einen Prototypen der zwar als Grundlage für den produktiven Einsatz verwendet werden kann jedoch noch viel Verbesserungspotential bietet. Mit einem auf den Page Object Generator perfekt abgestimmten Testharness lässt sich der Konfigurationsaufwand des Generators auf nahezu null reduzieren. Das senkt die Hämschwelle für die Benutzung des Tools und steigert die Akzeptanz beim Benutzer.
Darüber hinaus lässt sich mit einem gut vorbereiteten Testharness die Verwendung von best practice Ansätzäen weiter unterstützt.
Ein weitere Richtung in die der Page Object Generator verbessert werden könnte, wäre es verschiedenen Templates für die Generierung der Page Objects anzubieten.
Die derzeit im Page Object Generator angebotenen Templates und damit die generierten Page Object Klassen haben sich vom Selenium vorgeschlagenen Standard für Page Objects entfernt. Selenium schlägt die Verwendung von Page Objects mit annotierten WebElements als variablen vor. Die mit Locatoren annotierten WebElements werden bei der Instanziierung über eine PageFactory-Klasse aufgelöst und befüllt.
Im Gegensatz dazu erzeugen die im Page Object Generator derzeit hinterlegten Templates PageObject-Klassen die an Stelle von WebElementes die Klasse Control verwenden und Abhängig von der Existent der Klasse ByFactory sind (siehe Kapitel \ref{sec:selenipotestharness}). Um die von Selenium angebotene PageFactory zu unterstützen müssen neue Templates geschaffen welche die von der PageFactory-Klasse benötigten Konventionen erfüllen.
Diese Templates hätten den Vorteil, dass der Page Object Generator auch außerhalb der Landeshauptstadt München, für die der Generator in erster Linie entwickelt wurde, leichter Verbreitung finden könnte. Die von den Templates aktuell streng vorgegebene Struktur für das Testprojekt ist bei erster Betrachtung nicht sofort verständlich da sie sich vom Standard entfernt hat und erschwert damit Benutzern welche an die Arbeit mit der PageFactory-Klasse gewohnt sind den Einstieg.












