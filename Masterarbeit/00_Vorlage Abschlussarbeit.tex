%Dokumentklasse
\documentclass[a4paper,12pt]{scrreprt}
\usepackage[left= 2.5cm,right = 2cm, bottom = 4 cm]{geometry}
 \usepackage[onehalfspacing]{setspace}
% ============= Packages =============



% Dokumentinformationen
\usepackage[
	pdftitle={Testautomatisierung},
	pdfsubject={},
	pdfauthor={Matthias Karl},
	pdfkeywords={}
	pdftex=true, 
	colorlinks=true,
 	breaklinks=true,
	citecolor=black,
	linkcolor=black,	
	menucolor=black,	
	urlcolor=black
]{hyperref}


% Standard Packages
\usepackage[backend=biber,style=alphabetic]{biblatex}
\addbibresource{Literatur.bib}
\usepackage[autostyle]{csquotes}
\usepackage[utf8]{inputenc}
\usepackage[ngerman]{babel}
\usepackage[T1]{fontenc}
\usepackage{graphicx}
\graphicspath{{img/}}
\usepackage{fancyhdr}
\usepackage{lmodern}
\usepackage{color}
\usepackage{transparent}
\usepackage{caption, booktabs}
\usepackage{listings}
\usepackage{subfigure}
\usepackage{enumitem}
\setcounter{secnumdepth}{3}
% \setcounter{tocdepth}{4} 
\usepackage{graphicx}
\usepackage{pdflscape}




%code-Colour
\definecolor{javared}{rgb}{0.6,0,0} % for strings
\definecolor{javagreen}{rgb}{0.25,0.5,0.35} % comments
\definecolor{javapurple}{rgb}{0.5,0,0.35} % keywords
\definecolor{javadocblue}{rgb}{0.25,0.35,0.75} % javadoc
\definecolor{gray}{rgb}{0.5,0.5,0.5}

\lstset{frame=tb,
  language=Java,
  aboveskip=3mm,
  belowskip=3mm,
  showstringspaces=false,
  columns=flexible,
  basicstyle={\small\ttfamily},
  numbers=left,
  numberstyle=\tiny\color{gray},
  stepnumber=1,
  keywordstyle=\color{javapurple}\bfseries,
  commentstyle=\color{javagreen},
  morecomment=[s][\color{javadocblue}]{/**}{*/},
  stringstyle=\color{blue},
  breaklines=true,
  breakatwhitespace=true,
  tabsize=3
}



% zusätzliche Schriftzeichen der American Mathematical Society
\usepackage{amsfonts}
\usepackage{amsmath}

% nicht einrücken nach Absatz
\setlength{\parindent}{0pt}


% ============= Kopf- und Fußzeile =============
\pagestyle{fancy}
%
\lhead{}
\chead{}
\rhead{\slshape \leftmark}
%%
\lfoot{}
\cfoot{}
\rfoot{\thepage}
%%
\renewcommand{\headrulewidth}{0.4pt}
\renewcommand{\footrulewidth}{0pt}

% ============= Package Einstellungen & Sonstiges ============= 

%Besondere Trennungen
\hyphenation{De-zi-mal-tren-nung St-rei-fen-licht-scan-nern}
\hyphenation{Desktop-anwendungen}
\hyphenation{Design-entscheidungen}
\hyphenation{Test-automatisierung}
\hyphenation{System-entwurf}
\hyphenation{Frame-works}
\hyphenation{Test-orakel}
\hyphenation{Funktion-alitäten}
\hyphenation{Page}
\hyphenation{PageObject}
\hyphenation{Testcode-erstellung}
\hyphenation{Regres-sions-tests}
\hyphenation{OHMAP}
\hyphenation{SeleniPo}
\hyphenation{ControlBys}
\hyphenation{WebElemente}
\hyphenation{Ausliefer-ung}
\hyphenation{Code}
\hyphenation{Grundkonfigura-tion}


%römische Aufzählungen mit \RM{Zahl}
\newcommand{\RM}[1]{\MakeUppercase{\romannumeral #1}}


% ============= Dokumentbeginn =============

\begin{document}

\pagestyle{empty}
\begin{center}
\begin{tabular}{p{\textwidth}}


\begin{center}
\includegraphics[scale=0.17]{img/logos.jpg}
\end{center}


\\

\begin{center}
\LARGE{\textsc{
Testautomatisierung mit Selenium \\  \large{- Teilautomatisierte Generierung von Page Objects -}
}}
\end{center}

\\


\begin{center}
\large{Fakultät für Informatik und Mathematik \\
der Hochschule München \\}
\end{center}

\\

\begin{center}
\textbf{\Large{Masterarbeit}}
\end{center}



\begin{center}
vorgelegt von
\end{center}

\begin{center}
\large{\textbf{Matthias Karl}} \\
\large{Matrikel-Nr: 03280712} \\
\end{center}

\begin{center}
\large{am 29.01.2016}
\end{center}

\\

\\

\begin{center}
\begin{tabular}{lll}
\textbf{Prüfer:} & & Prof. Dr. Ullrich Hafner\\
\textbf{Zweitprüfer:} & &  Prof. Dr. Oliver Braun\\
\end{tabular}
\end{center}

\end{tabular}
\end{center}

% \part im Inhaltsverzeichnis nicht nummerieren
\makeatletter
\let\partbackup\l@part
\renewcommand*\l@part[2]{\partbackup{#1}{}}

%Seitennummerierung neu beginnen, Zahlen [arabic], röm.Zahlen [roman,Roman], Buchstaben [alph,Alph]
\pagenumbering{Roman}

\newpage
\include{02_danksagungen}

\addsec{Zusammenfassung / Abstract}
\label{sec:zusammenfassung}
Diese Arbeit befasst sich mit dem Themenschwerpunkt Testautomatisierung.
Zu Beginn werden zunächst die Grundlagen im Bereich der Software-Qualität und dem Testen im allgemeinen geschaffen, um dann einen Überblick über die verschiedenen Möglichkeiten der Testautomatisierung zu präsentieren.
Anhand der geschaffenen Grundlagen wird das Testautomatisierungs-Tool Selenium vorgestellt. Hierbei liegt ein besonderes Augenmerk auf dem Page Object Design Pattern. Bezüglich des Page Object Design Pattern wird eine selbst erstellte Software-Lösung vorgestellt, welche die Verwendung dieses Pattern durch eine teilautomatisierte Generierung von Page Object-Klassen vereinfachen soll.

\newpage
\pagestyle{fancy}
%Inhaltsverzeichnis
\tableofcontents

%Verzeichnis aller Bilder
\newpage
\listoffigures

\newpage
%Seitennummerierung neu beginnen, Zahlen [arabic], röm.Zahlen [roman,Roman], Buchstaben [alph,Alph]
\pagenumbering{arabic}
% pagestyle für gesamtes Dokument aktivieren
\pagestyle{fancy}

\newpage
\chapter{Einleitung}
\label{sec:einleitung}
Software hat in der heutigen Zeit für Unternehmen und Privatpersonen an großer Bedeutung gewonnen.
Die Komplexität der Softwareprodukte steigt stetig an, gleichzeitig auch die Anforderungen an die Qualität.
Das hat zur Folge, dass bei der Entwicklung von Software ein immer größeres Augenmerk auf den Bereich des Testens gelegt wird.
Vor allem die Testautomatisierung rückt hierbei immer mehr in den Vordergrund.
Man verspricht sich dadurch eine stetige Qualitäts- und Effizienzsteigerung.\\
Eine Vielzahl der gängigen Software wird heutzutage in Form von Webanwendungen genutzt.
Um diese automatisiert zu Testen, wird oft auf Testfälle zurückgegriffen, die automatisch Eingaben auf der Oberfläche der Anwendung tätigen, um anschließend das Verhalten der Anwendung zu überprüfen.
Ein gängiges Tool, mit welchem diese Form der Testfälle umgesetzt werden kann, ist Selenium \cite{selenium_selenium_2015}. Selenium ist ein Tool, welches auch bei it@M, dem externen IT-Dienstleister der Landeshauptstadt München, für automatisierte Softwaretests eingesetzt wird. Im Rahmen dieser Arbeit soll eine Möglichkeit aufgezeigt werden, wie die Verwendung dieses Tools bei der Landeshauptstadt München einfacher und effizienter gestaltet werden kann.
\\

\section{Motivation}
\label{sec:motivation}
Testautomatisierung hat oft das Problem, dass Testfälle zwar wiederholt und einfach ausgeführt werden können, der initiale Mehraufwand für die Erstellung der Testfälle ist aber, verglichen mit manuellen Tests, häufig so hoch, dass die erreichten Vorteile nur gering zum Tragen kommen. Ein möglicher Weg, um Verbesserungen in der Testautomatisierung zu erzielen, ist es daher, diesen Mehraufwand zu minimieren.\\
Testfälle, die mit Hilfe des Selenium WebDiver entwickelt werden, leiden oftmals unter eben dieser Schwierigkeit. Die initiale Erstellung der Tests ist in der Regel sehr aufwändig.
Das liegt unter anderem daran, dass bei der Verwendung des WebDrivers meist ein bestimmtes Design Pattern, das Page Object Pattern, verwendet wird. Von diesem Design Pattern verspricht man sich möglichst wartbare und stabile Testfälle.
Bestandteil des Pattern ist es, eine Vielzahl von sogenannten Page Objects zu erstellen, welche die einzelnen Seiten einer Webanwendung repräsentieren.\\
In Hinblick auf den initialen Mehraufwand weist die Erstellung dieser Page Object-Klassen ein großes Einsparungspotential auf. Aufgrund ihrer generischen Struktur bieten die Page Objects nämlich das Potential automatisch erzeugt zu werden.
In Zusammenarbeit mit dem IT-Dienstleister der Landeshauptstadt München (it@M) soll daher eine Software entwickelt werden, mit deren Hilfe die Erstellung der Page Object-Klassen vereinfacht werden kann.

\section{Roadmap}
\label{roadmap}
Der Hauptteil dieser Arbeit gliedert sich in vier Kapitel. In Kapitel \ref{sec:grundlagen} werden zunächst die Grundlagen in den Bereichen der Software-Qualität, dem Testen im Allgemeinen und der Testautomatisierung im Speziellen gelegt.
Kapitel \ref{sec:testautomatisierung} geht näher auf die Testautomatisierung ein und soll einen Überblick über die verschiedenen Bereiche und Möglichkeiten geben, welche die Testautomatisierung bietet.
Kapitel \ref{sec:testautomatisierung_mit_selenium} beschäftigt sich mit dem Testautomatisierungstool Selenium und erläutert in diesem Zusammenhang das Page Object Pattern.
In Kapitel \ref{sec:teilautomatisierte_generierung_von_pageObjects} wird eine Softwarelösung vorgestellt mit deren Hilfe die Verwendung des Page Object Pattern unterstützt werden kann.

  





\newpage
\chapter{Grundlagen}
\label{sec:grundlagen}


\section{Software-Qualität}
\label{sec:softwarequalität}

Nahezu jeder Programmierer ist schon einmal mit dem Begriff der Software-Qualität in Berührung gekommen. Diesen Qualitätsbegriff jedoch genau zu fassen erweist sich als schwierig.
Die DIN-ISO-Norm 9126 definiert Software-Qualität wie folgt:
\\
\glqq Software-Qualität ist die Gesamtheit der Merkmale und Merkmalswerte eines Software-Produkts, die sich auf dessen Eignung beziehen, festgelegte Erfordernisse zu erfüllen.\grqq \cite{iso/iec_iso/iec_2001}
\\
Nach Hoffmann \cite[vgl. S.6 ff.]{hoffmann_software-qualitat_2013} wird aus dieser Definition deutlich, dass es sich bei dem Begriff der Software-Qualität um eine multikausale Größe handelt. Das bedeutet, dass zur Bestimmung der Qualität einer Software nicht ein einzelnes Kriterium existiert. Vielmehr verbergen sich hinter dem Begriff eine ganze Reihe verschiedener Kriterien, die je nach den gestellten Anforderungen in ihrer Relevanz variieren.
Sammlungen solcher Kriterien werden in sogenannten Qualitätsmodellen zusammengefasst. Die DIN-ISO-Norm 9126 \cite{iso/iec_iso/iec_2001} bietet ein solches Qualitätsmodell und definiert damit eine Reihe von wesentlichen Merkmalen, die für die Beurteilung der Software-Qualität eine Rolle spielen. Diese Merkmale sind in der Abbildung \ref{fig:qualitaetsmerkmaleVonSoftwaresystemen} zusammengefasst.
\begin{figure}[htb]
  \centering  
  \includegraphics[scale=0.6]{img/softwarequalitaet9126.png}\\
  \footnotesize\sffamily\textbf{Quelle:} \cite{iso/iec_iso/iec_2001}
  \caption{Qualitätsmerkmale von Softwaresystemen (ISO 9126)}
  \label{fig:qualitaetsmerkmaleVonSoftwaresystemen}
\end{figure}
Eine nähere Definition der einzelnen Begriffe des Qualitätsmodells kann beispielsweise dem Buch Software-Qualität von Dirk W. Hoffmann \cite[S.7 ff.]{hoffmann_software-qualitat_2013} entnommen werden. 
Um die Qualität einer Software zu Steigern, bietet die moderne Software-Qualitätssicherung laut Hofmann \cite[vgl. S.19 ff.]{hoffmann_software-qualitat_2013} eine Vielzahl von Methoden und Techniken:
Ein Teil der Methoden versucht durch eine Verbesserung des Prozesses der Produkterstellung die Entstehung von qualitativ hochwertigen Produkten zu begünstigen. Diese Methoden fallen in den Bereich der Prozessqualität.
Einen weiteren Bereich bilden die Methoden die zur Verbesserung der Produktqualität dienen. Bei diesen Methoden wird das Softwareprodukt direkt bezüglich der Qualitätsmerkmale überprüft. Dieser Bereich unterteilt sich in die konstruktive und analytische Qualitätssicherung. Unter konstruktiver Qualitätssicherung versteht man den Einsatz von z.B. Methoden, Werkzeugen oder Standards die
dafür sorgen, dass ein Produkt bestimmte Forderungen erfüllt. 
Unter analytische Qualitätssicherung versteht man den Einsatz von analysierenden bzw. prüfenden Verfahren, die Aussagen
über die Qualität eines Produkts machen.
In diesem Bereich der Qualitätssicherung befindet sich unter anderem der klassische Software-Test. Eine Übersicht über das gesamte Gebiet der Software-Qualitätssicherung, wie es sich uns gegenwärtig darstellt, ist in Abbildung \ref{fig:softwareQualitätssicherung} dargestellt. 
\begin{figure}[htb]
  \centering  
  \includegraphics[scale=0.7]{img/softwarequalitaet.png}\\
  \footnotesize\sffamily\textbf{Quelle:} \cite[vgl. S.20]{hoffmann_software-qualitat_2013}
  \caption{Übersicht über das Gebiet der Software-Qualitätssicherung}
  \label{fig:softwareQualitätssicherung}
\end{figure}



\section{Softwaretest}
\label{sec:softwaretest}
Im laufe der Zeit wurden viele Versuche unternommen um die Qualität von Software zu steigern. Besondere Bedeutung hat dabei der Software-Test erlangt.
Der IEEE Std 610.12 definiert den Begriff Test als das ausführen einer Software unter bestimmten Bedingungen mit dem Ziel, die erhaltenen Ergebnisse auszuwerten, also gegen erwartete Werte zu vergleichen.
(Im Original: \glqq An activity in which a system or component is executed under specific conditions, the results are observed or recorded, and an evaluation is made of some aspect of the system or component.\grqq\ \cite{ieee_ieee_1991})
Bereits zu beginn der Softwareentwicklung hat man versucht Programme vor ihrer Auslieferung zu testen. Der dabei erzielte Erfolg entsprach nicht immer den Erwartungen. Im laufe der Jahre wurde das Testen daher auf eine immer breitere Grundlage gestellt. Es entwickelten sich Unterteilungen des Software-Tests die bis heute bestand haben. Thaller \cite[vgl. S.18]{thaller_software-test_2002}  nennt hier beispielsweise:
\begin{itemize}
\item White-Box-Test
\item Black-Box-Test und externe Testgruppe
\item Volume Test, Stress Test und Test auf Systemebene
\end{itemize}
Jeder dieser Begriffe beschreibt bestimmte Techniken, die bei konsequenter Anwendung dazu führen können Fehler in Softwareprodukten zu identifizieren. 

Nach Hoffmann \cite[vgl. S.22]{hoffmann_software-qualitat_2013} spielt neben der Auswahl der richtigen Techniken für ein bestimmtes Problem in der Praxis die Testkonstruktion eine zentrale Rolle. Bereits für kleine Programme ist es faktisch nicht mehr möglich das Verhalten einer Software für alle möglichen Eingaben zu überprüfen. Es muss sich daher immer auf eine vergleichsweise winzige Auswahl an Testfällen beschränkt werden. Testfälle unterscheiden sich jedoch stark in ihrer Relevanz. Die Auswahl der Testfälle hat daher einen großen Einfluss auf die Anzahl der gefundenen Fahler und damit auch auf die Qualität des Endprodukts. 

Dennoch ist der Software-Test laut Hofmann \cite[vgl. S.22]{hoffmann_software-qualitat_2013} eines der am meisten verbreiteten Techniken zur Verbesserung der Softwarequalität. Um über lange Sicht gute Software zu produzieren reicht es jedoch nicht aus sich nur auf diese Technik allein zu stützen. Ein großer Nachteil des Softwaretests ist laut Thaller \cite[vgl. S.18]{thaller_software-test_2002}, dass Fehler erst in einer relativ späten Phase der Entwicklung identifiziert werden. Je später eine Fehler jedoch erkannt wird, desto teurer wird auch seine Beseitigung. Abbildung \ref{fig:softwareQualitätssicherung} zeigt, dass der Software-Test nur eine von vielen Techniken des Qualitätsmanagement darstellt. Um eine möglichst qualitativ hochwertige Software zu erhalten, ist es daher ratsam, sich bei der Qualitätssicherung möglichst breit aufzustellen und sich nicht nur auf die analytische Qualitätssicherung in Form des Software-Tests zu verlassen. 


\section{Testautomatisierung}
\label{sec:testautoGrundlagen}
Das Testen von Software macht in heutigen Projekten einen großen Teil der Projektkosten aus. So sprechen beispielsweise Harrold \cite{harrold_testing:_2000} und auch Ramler \cite{ramler_economic_2006} davon, dass das Testen für 50\% 
und mehr der gesamten Projektkosten verantwortlich sein kann. 
Mit Steigender Komplexität der Softwäre steigen diese Kosten immer weiter an.  
Um diese Kosten zu reduzieren haben sich im laufe der Zeit die bestehenden Testmethoden immer weiter entwickelt und auch neue Ansätze herausgebildet. Harrold \cite{harrold_testing:_2000} beschreibt als einen Ansatz, Software-Tests möglichst automatisiert durchzuführen. Diesen Ansatz fast man mit dem Begriff Testautomatisierung zusammen.
Seidl et al. \cite[S.7]{seidl_basiswissen_2012} definieren Testautomatisierung als \glqq die Durchführung von ansonsten manuellen Testtätigkeiten durch Automaten.\grqq
Diese Definition zeigt, dass das Spektrum der Testautomatisierung breit gefächert ist. Testautomatisierung beschränkt sich nicht nur auf das automatisierte durchführen von Testfällen sondern erstreckt sich über alle Bereiche des Software-Test. Die Verschiedenen Möglichkeiten der Testaustomatisierung werden in Kapitel \ref{sec:bereiche_der_testautomatisierung} dargestellt.
Aus Sicht des Qualitätsmanagement ist die Testautomatisierung sowohl den Methoden zur Steigerung der Produktqualität als auch der Prozessqualität zugeordnet. Ein automatisierter Software-Test hat immer noch den selben Charakter wie ein manueller Software-Test und ist daher ein Teil der analytischen Qualitätssicherung. Allerdings erfordert Testautomatisierung laut Hoffmann \cite[vgl. Seite 25]{hoffmann_software-qualitat_2013} auch immer infrastrukturelle Anpassungen. Automatisierte Testfälle benötigen in der Regel eine Besondere Software-Infrastruktur wie etwa ein Automatisierungsframework. Solche Maßnahmen, die den Programmentwickler aus technischer Sicht in die Lage versetzen, seiner täglichen Arbeit in geregelter und vor allem produktiver Weise nachzugehen, werden den Methoden zur Verbesserung der Prozessqualität zugeordnet. (siehe Abbildung \ref{fig:softwareQualitätssicherung}).



\section{Testprozess}
\label{sec:testprozess}

Um Software-Tests effektiv und Strukturiert durchzuführen wird eine verfeinerter Ablaufplan für die einzelnen Testaufgaben benötigt. Diesen Ablaufplan fassen Splinner und Linz \cite{spillner_basiswissen_2007} im fundamentalen Testprozess zusammen.  Die einzelnen Arbeitsschritte die im Lebenszyklus eines Software-Tests anfallen werden dabei verschiedenen Phasen zugeordnet.
Durch den Testprozess wird die Aufgabe des Testens so in kleinere Testaufgaben untergliedert.

Testaufgaben, die man dabei unterscheidet sind:

\begin{itemize}
	  \itemsep0pt
      \item Testplanung und Steuerung
      \item Testanalyse und Testdesign
      \item Testrealisierung und Testdurchführung
      \item Testauswertung und Bericht
      \item Abschluss der Testaktivitäten       
\end{itemize}

\glqq Wer strukturiert testet, wird, unabhängig vom jeweiligen Vorgehen, diese Aktivitäten auf die eine oder andere Weise abbilden.\grqq\ \cite[S. 9]{seidl_basiswissen_2012} \\
\glqq Obgleich die Aufgaben in sequenzieller Reihenfolge im Testprozess angegeben sind, können sie sich überschneiden und teilweise auch gleichzeitig durchgeführt werden.\grqq\ \cite[S.19]{spillner_basiswissen_2007} \\ Auf Grundlage des fundamentalen Testprozesses nach Splinner und Linz \cite[S.20ff]{spillner_basiswissen_2007} werden im folgenden diese Teilaufgaben näher beschrieben. 
Diese Beschreibung wird durch Ausführungen von Seidl et al. \cite[S. 9 ff.]{seidl_basiswissen_2012} erweitert. 

\subsection{Testplanung und Steuerung}
\label{subsec:testplanung_und_steuerung}
Um dem Umfang und der Komplexität heutiger Software-Tests gerecht zu werden, benötigt man zu Beginn des Testprozesses eine genaue Planung.
Ziel dieser Planung ist es, den Rahmen für die weiteren Testaktivitäten festzulegen. Die Aufgaben und die Zielsetzungen der Tests müssen ermittelt wurden. Eine Ressourcenplanung wird benötigt und eine geeignete Teststrategie muss ermittelt werden. In Kapitel \ref{sec:softwaretest} wurde bereits erwähnt, dass das vollständige Testen einer Anwendung in der Regel nicht möglich ist. Die einzelnen Systemteile müssen daher nach Schwere der zu erwartenden Fehlerwirkung priorisiert werden. Um so schwerwiegender die zu erwartende Fehlerwirkung ist, umso intensiver muss der betrachtete Systemteil auch getestet werden. Ziel der Teststrategie ist also \glqq die optimale Verteilung der Tests auf die \frqq richtigen\flqq\ Stellen das Softwaresystems.\grqq\ \cite[S.21]{spillner_basiswissen_2007} \\ Steht das Softwareprojekt unter einem hohen Zeitdruck müssen Testfälle zusätzlich Priorisiert werden.
Um zu verhindern, dass das Testen zu einem endlosen Prozess wird, müssen auch geeignete Testendekriterien festgelegt werden. Anhand dieser Kriterien kann später entschieden werden ob der Testprozess abgeschlossen werden kann.

Bereits zu Beginn des Testprozesses werden auch wichtige Grundsteine für eine spätere Testautomatisierung gelegt. Es muss entschieden werden, in welchen Teststufen und Testbereichen eine Automatisierung eingesetzt werden soll. Vor allem ist die frage zu klären ob und in welchem Ausmaß eine Automatisierung überhaupt sinnvoll ist. Es kann durchaus vorkommen, dass eine Analyse ergibt, dass eine Testautomatisierung für ein Projekt unwirtschaftlich ist.
Entscheidet man sich für eine Testautomatisierung hat das in der Regel große Auswirkung auf die einzusetzenden Ressourcen und die Zeitliche Planung und Aufwandsschätzung.
Oftmals wird im Rahmen der Tests eine besondere Werkzeugunterstützung oder Infrastruktur benötigt. Derartige Punkte müssen auch bereits in der frühen Planungsphase berücksichtigt werden.

Die Gesamten erarbeiteten Rahmenbedingungen werden in einem Testkonzept dokumentiert.
Eine mögliche Vorlage für dieses Dokument bietet die internationale Norm IEEE 829-2008 \cite{ieee_ieee_2008}.
Neben der frühzeitigen Planung der Tests muss während des gesamten Testprozesses eine Steuerung erfolgen.
Hierfür werden die Ergebnisse und Fortschritte der Tests und des Projekts laufend erhoben, geprüft und bewertet. Werden Probleme erkannt, kann so rechtzeitig gegengesteuert werden. 

\subsection{Testanalyse und Testdesign}
\label{subsec:testanalyse_und_design}
In dieser Phase wird zunächst die Qualität der Testbasis überprüft. Alle Dokumente die für die Erstellung der Testfälle benötigt werden müssen in ausreichendem Detailgrad vorhanden sein. Mit Hilfe der qualitätsgesicherten Dokumente kann die eigentliche Testfallerstellung beginnen.
Anhand der Informationen aus dem Testkonzept und den Spezifikationen werden mittels strukturierter Testfallerstellungsmethoden logische Testfälle erstellt. Diese logischen Testfälle können dann in einer späteren Phase konkretisiert werden, indem ihnen z.B. tatsächliche Eingabewerte zugeordnet werden. Für jeden dieser Testfälle müssen die möglichen Rand- und Vorbedingungen so wie ein erwartetes Ergebnis bestimmt werden. 
In dieser Phase beginnt auch die Umsetzung der Testfälle in der Testautomatisierung.
Abgestimmt auf die ausgewählten Automatisierungswerkzeuge und die zu testende Software muss die Umgebung für die Testautomatisierung vorbereitet werden. Anhand der Vorgaben des Testkonzeptes können dann jene Testfälle und Testabläufe ausgewählt werden die im Zuge der Testautomatisierung implementiert werden sollen. Hierbei wird noch einmal die technische Umsetzbarkeit der ausgewählten Testfälle geprüft. Bei der Auswahl der Testfälle sollte zu beginn eine möglichst breite Testabdeckung angestrebt werden.
Problemfelder können dann später durch weitere Testfälle in in der Tiefe getestet werden.

\subsection{Testrealisierung und Testdurchführung}
\label{subsec:testrealisierung_und_durchführung}
In diesem Schritt des Testprozesses werden aus den Logischen Testfällen der vorangegangenen Phase konkrete Testfälle gebildet.
Diese Testfälle werden anhand ihrer fachlichen und technischen Zusammengehörigkeit zu Testszenarien gruppiert und anhand der Vorgaben aus dem Testkonzept priorisiert.
Sobald die zu testende Software zur Verfügung steht, kann mit der Abarbeitung der Testfälle begonnen werden. Die dabei erhaltenen Ergebnisse werden vollständig protokolliert. Werden im Zuge der Durchführung Fehler aufgedeckt, muss darauf in geeigneter Weise reagiert werden. Es könnte beispielsweise ein zuvor definierter Fehlerprozess gestartet werden.
Korrekturen und nachgehende Veränderungen am Testobjekt werden durch eine Wiederholung der Testläufe abgedeckt.
Aus Sicht der Testautomatisierung beginnt in dieser Phase die technische Umsetzung der Testfälle.
In vielen Fällen bedeutet das Programmiertätigkeit. Diese Programmiertätigkeiten sind wiederum anfällig für eigene Fehler und müssen daher in angemessener Weise selbst qualitätsgesichert werden. Auch bei der Testautomatisierung ist eine Zusammenfassung von Testfällen sinnvoll. Auf diese weise kann man funktionalen und logischen Abhängigkeiten zwischen den Testfällen gerecht werden.
Nach der Implementierung können die geplanten Testfälle durchgeführt werden.
Gerade bei der Automatisierung ist eine genaue Protokollierung der Ergebnisse besonders wichtig.
Nur dadurch ist es später möglich, aufgetretene Fehler überhaupt zu lokalisieren.


\subsection{Testauswertung und Bericht}
\label{subsec:testauswertung_und_bericht}
In dieser Phase des Prozesses wird geprüft, ob die im Testkonzept definierten Testendekriterien erreicht wurden. Sind alle Forderungen erfüllt, kann es zu einem Abschluss der Testaktivitäten kommen. Kommt es zu Abweichungen im Bezug auf diese Kriterien, muss darauf entsprechend reagiert werden. Es können Fehlerkorrekturen durchgeführt werden oder neue Testfälle erstellt werden. Aber auch der umgekehrte Fall ist möglich. Es kann dazu kommen, dass Endekriterien nur mit unverhältnismäßig hohem Aufwand erreicht werden 
und daher bestimmte Testfälle entfallen oder Kriterien überdacht werden müssen.

Für die Testautomatisierung ist die wesentliche Aufgabe dieser Phase die Auswertung und Aufarbeitung der erhaltenen Ergebnisse. Automatisierte Tests generieren oftmals eine Fülle an Log-Dateien und Protokollen. Um aus diesen Ergebnissen die richtigen Schlüsse zu ziehen und sie für dritte zugänglich zu machen, müssen sie in eine lesbare Form gebracht werden.

In jedem Fall muss über die erhaltenen Ergebnisse und das daraus resultierende Vorgehen ein Testbericht erstellt werden. Je nach Umfang und Phase der Test kann dieser mehr oder weniger formal ausfallen. Für einen Komponententest reicht beispielsweise eine formlose Mitteilung. Höhere Teststufen erfordern allerdings einen formaleren Bericht.



\subsection{Abschluss der Testaktivitäten}
\label{subsec:abschluss_der_testaktivitäten}
Sind die Testaktivitäten beendet, sollten zum Schluss alle im laufe des Testprozesses gemachten Erfahrungen analysiert werden. So können die gewonnenen Erkenntnisse für spätere Projekte genutzt werden. Dadurch kann eine stetige Verbesserung des Testprozesses erreicht werden.
Die während des Prozesses erstellte Testware sollte archiviert werden. Auf diese weise steht sie für folgende Regressionstests zur Verfügung. Die Kosten in Wartung und Pflege der Software können damit gesenkt werden.
Bei der Testautomatisierung bedeutet das, die Wiederverstellbarkeit der Testumgebung und des Sourcecodes der Tests sicherzustellen.
\newline\\
Abschließend ist zu sagen, dass sich die Testautomatisierung in der Regel gut in einen bereits bestehenden Testprozess integrieren lässt. Sie wird allerdings \glqq den Prozess nicht verbessern oder gerade richten, sondern nur unterstützen.\grqq\ \cite[S.21]{seidl_basiswissen_2012} \\ Ist der Testprozess schon vor Einführung einer Automatisierung schlecht gelaufen, wird er sich nach der Einführung nicht verbessern.
Die Testautomatisierung ist also nicht als Heilmittel für schlecht laufende Prozesse gedacht sondern als Möglichkeit einen bereits gut laufenden Prozess effizienter zu gestalten.

\section{Vorgehensmodelle}
\label{sec:vorgehensmodelle}
Der in Kapitel \ref{sec:testprozess} beschriebene Testprozess ist nicht als losgelöster, eigenständiger Prozess zu betrachten. Vielmehr ist der Testprozess immer ein Teil eines größeren Entwicklungsablaufes bei der Erstellung eines Softwareprodukts. Einen solchen Entwicklungsablauf versucht man mit Hilfe von sogenannten Softwareentwicklungsmodellen, auch Vorgehensmodelle genannt, abzubilden.
Ein Projekt wird dazu in einzelne Phasen untergliedert an deren Ende ein gewisses Ziel bzw. Ergebnis steht.
Auf gröbster Ebene lassen sich die Abläufe auf vier Hauptphasen reduzieren. Diese Phasen finden sich mehr oder weniger Ausgeprägt in den meisten der gängigen Vorgehensmodelle wieder und werden auch so von Seidl et al. \cite[S.21 ff.]{seidl_basiswissen_2012} verwendet.:

\begin{itemize}
\item Spezifikation
\item Design
\item Entwicklung
\item Test
\end{itemize}

Das Testen, bzw. der Testprozess ist eine von mehreren Phasen in solch einem Entwicklungsmodell.
Es gibt eine Vielzahl von unterschiedlichen Softwareentwicklungsmodellen. Der Hauptunterschied liegt meist in der zeitlichen Koppelung und der inhaltlichen Ausprägung der einzelnen Phasen. Die einzelnen Phasen können sich innerhalb eines Vorgehensmodells überschneiden und wiederholen und müssen auch nicht immer wie in der Auflistung angegeben sequentiell abgearbeitet werden.
Aus der Sicht der Testautomatisierung ist nach Seidl et al. \cite[vgl. S.21 ff.]{seidl_basiswissen_2012} eine Einteilung der verschiedenen Vorgehensmodelle in zwei Gruppen sinnvoll: 

\begin{itemize}
\item Klassische Entwicklungsmodelle, die eher sequentiell ausgerichtet sind
\item Iterative und agile Entwicklungsmodelle, die sich durch Parallelisierung und kurze Iterationen auszeichnen.
\end{itemize}

\subsection{Klassische Entwicklungsmodelle}
\label{subsec:klassische_entwicklungsmodelle}

Die hier als Klassische Entwicklungsmodelle betitelten Vorgehensmodelle zeichnen sich vor allem dadurch aus, dass die einzelnen Phasen sequentiell ausgeführt werden. Der bekannteste Vertreter dieser Vorgehensmodelle ist das Wasserfallmodell \cite{royce_managing_1987}. In diesem Modell sind alle Phasen strickt voneinander getrennt. Eine neue Phase kann erst begonnen werden, wenn eine vorangegangene Phase abgeschlossen wurde. Rücksprünge in vorangegangenen Phasen sind unerwünscht. In der Praxis wird diese vorgehen laut Seidl et al. \cite[vgl. S.22]{seidl_basiswissen_2012} jedoch oft nicht ganz so strickt umgesetzt. Es kommt zu Mischformen, bei denen die einzelnen Phasen nicht mehr voll sequentiell abgearbeitet werden sondern sich teilweise überlagern. Vor allem im Bereich des Testens geht man oft zu einer solchen Überlagerung über. Das Testen ist meist keine getrennte Phase am Ende des Entwicklungsprozesses sondern erstreckt sich über den gesamten Prozess ausgehend von der frühen Spezifikationsphase.

\begin{figure}[htb]
  \centering  
  \includegraphics[scale=0.7]{img/sequentielleentwicklungsmodelle.png}\\
  \footnotesize\sffamily\textbf{Quelle:} \cite[vgl. S.22]{seidl_basiswissen_2012}
  \caption{Verschiedene Ausprägungen klassischer Entwicklungsmodelle}
  \label{fig:verschiedene_auspraegungen_klassischer_entwicklungsmodelle}
\end{figure}

Seidl et al. \cite[vgl. S.22]{seidl_basiswissen_2012} stellen fest, dass in Projekten die ein solch sequentielles Vorgehen wählen, bereits in der frühen Planungsphase des Testprozesses genau abgewägt werden muss, ob eine Automatisierung der Testfälle überhaupt sinnvoll ist.
Wenn zu Beginn des Projektes schon klar ist, dass die Testfälle nur ein einziges mal am Ende des Entwicklungsprozesses ausgeführt werden, steht eine Automatisierung oft nicht in Relation zu den erhöhten Kosten die bei der Erstellung der Testfälle anfallen würden.
Bei dieser Entscheidung ist allerdings zu beachten, dass Software meist mit dem Ende eines Projektes nicht seinen finalen Stand erreicht hat. Fehler so wie geänderte Anforderungen führen meist dazu, dass sich Softwareprodukte ständig Weiterentwickeln.
Diese Weiterentwicklung ist zwangsläufig mit Codeänderungen verbunden die wiederum zu Fehlern in bereits bestehenden Code führen kann.
Um solche Fehler zu entdecken müssen im Rahmen von Regressionstests auch Testfälle wiederholt werden die bereits erfolgreich abgeschlossen wurden.
Solche Regressionstests lassen sich bei einer vorhandenen Testautomatisierung besonders leicht durchführen. Sind also in der Software nach Projektabschluss größere Änderungen zu erwarten kann sich eine Automatisierung über längere Sicht durchaus lohnen.
Neben Regressionstests kann nach Seidl et al. \cite[vgl. S.23]{seidl_basiswissen_2012} auch die Notwendigkeit einer höheren Testtiefe oder einer breiteren Testabdeckung ein Faktor sein sich für eine Automatisierung zu entscheiden.
In manchen Fällen, wie beispielsweise Lasttests mit mehreren hundert Usern, kann eine Automatisierung auch unabdingbar werden.

\subsection{Iterative und agile Entwicklungsmodelle}
\label{subsec:iterative_und_agile_entwicklungsmodelle}
Als weiter Gruppe der Vorgehensmodelle nennen Seidl et al. \cite[vgl. S.23 ff.]{seidl_basiswissen_2012} die Iterative und agile Entwicklungsmodelle.\\
Im Gegensatz zu den klassischen Entwicklungsmodellen sind in iterativen Modellen Rücksprünge in vorangegangene Phasen explizit erlaubt. Eine oder alle Phasen werden in diesen Modellen wiederholt durchlaufen. Auf diese Weise kann das Softwareprodukt inkrementell wachsen. Durch ein derartiges Vorgehen ist es einfacher möglich auf den Umstand zu reagieren, dass sich Anforderungen in Softwareprojekten häufig ändern. Auch agile Vorgehensmodelle leben von solch einem iterativen Vorgehen. Die einzelnen Phasen werden in kleinen Zyklen viele male durchlaufen.
Ein bekannter Vertreter der agilen Methoden ist Scrum \cite{schwaber_agile_2002}. In Scrum wird ein Softwareprodukt in kurzen sogenannten Sprints realisiert. Innerhalb eines solchen Sprints wählt das Team selbständig eine Teilaufgabe des Projekts aus. Diese Teilaufgabe wird spezifiziert, designet, entwickelt und getestet. Am Ende eines Sprints steht ein Softwareprodukt, welches um ein weiteren Baustein ergänzt wurde.
Der Sprint ist das zentrale Element dieses Prozessmodelles und kennzeichnet eine Iteration.
\begin{figure}[htb]
  \centering  
  \includegraphics[scale=0.8]{img/iterativeentwicklungsmodelle.png}\\
  \footnotesize\sffamily\textbf{Quelle:} \cite[vgl. S.24]{seidl_basiswissen_2012}
  \caption{Verschiedene Ausprägungen iterativer und agiler Entwicklungsmodelle}
  \label{fig:verschiedene_auspraegungen_iterativer_und_agiler_entwicklungsmodelle}
\end{figure}
Für das Testen stellen diese kurzen Iterationen laut  Seidl et al. \cite[vgl. S.24]{seidl_basiswissen_2012} ein Problem dar.
Jeder Entwicklungszyklus bringt neue Features hervor die mit Testfällen abgedeckt werden müssen. Der Agile Charakter in diesen Vorgehensmodellen bedingt, dass sich Anforderungen ständig ändern und somit auch bereits fertiger Code oft angepasst werden muss. Darüber hinaus ist nicht ausgeschlossen, dass neue Features Auswirkungen auf alten Code haben können. Neben den neu implementierten Teilen muss daher zum Ende einer jeden Iteration auch sämtlicher alter Code getestet werden.
Das bedingt einen enormen Testaufwand am Ende einer jeden Iteration. 
In agilen Vorgehensmodellen wie Scrum ist der Testaufwand nach wenigen Sprints bereits so hoch, dass ein Testdurchlauf zusammen mit allen Regressionstests nicht mehr zu bewältigen ist.
Gerade in Projekten, die einem derartigen Vorgehensmodell folgen, ist es daher sinnvoll Testautomatisierung einzusetzen. Einmal implementierte Testfälle können zum Ende einer jeden Iteration erneut ausgeführt werden. Die höheren Kosten, die bei der Automatisierung entstehen sind so schnell amortisiert.\\
Das sich ständig ändernde Testobjekt bedingt nicht nur die Notwendigkeit von automatisierten Testfällen, es erhöht gleichzeitig auch die Anforderungen an die Qualität der Testfälle. Häufige Änderungen am zu testenden Code lassen einmal implementierte Testfälle schnell veralten. Es muss daher bei der Erstellung der automatisierten Tests besonders auf die Wartbarkeit geachtet werden. Testfälle sollten möglichst Robust gewählt werden und nicht schon durch kleine Änderungen am Testobjekt zerstört werden. Änderungen am Testobjekt sind in agilen Projekten unvermeidbar. Unter diesem Gesichtspunkt sollte daher auch das Design der Testfälle erfolgen. Automatisierte Testfälle sollten ähnliche Qualitätsstandards wie der Code des eigentlichen Projektes verfolgen. Anpassungen an den Testfällen werden sonst schnell zu zeitaufwändig. Die Pflege der bereits implementierten Testfälle wird dann nicht mehr tragbar und die Akzeptanz der Tests im Projekt sinkt.

\newpage
\chapter{Testautomatisierung}
\label{sec:testautomatisierung}
In Kapitel \ref{sec:testautoGrundlagen} wurde der Begriff Testautomatisierung bereits eingeführt. Die darin dargelegte Definition hat gezeigt, dass man unter Testautomatisierung nicht nur das automatisierte Ausführen von Testfällen versteht.
Testautomatisierung ist in allen Bereichen des Entwicklungs- bzw. Testprozesses möglich.
\glqq Das Spektrum umfasst alle Tätigkeiten zur Überprüfung der Softwarequalität im Entwicklungsprozess, in den unterschiedlichen Entwicklungsphasen und Teststufen sowie die entsprechenden Aktivitäten von Entwicklern, Testern, Analytikern oder auch der in die Entwicklung eingebundenen Anwender. Die Grenzen der Automatisierung liegen darin, dass diese nur die manuellen Tätigkeiten eines Testers übernehmen kann, nicht aber die intellektuelle, krative und intuitive Dimension dieser Rolle.\grqq\ \cite[S.7]{seidl_basiswissen_2012} \\
Die intellektuelle Dimension ist vor allem in den frühen Phasen des Testprozesses gefordert. Diese Phasen sind maßgeblich für die spätere Qualität der einzelnen Testfälle. Testautomatisierung wird daher nie die Arbeit eines Testanalysten voll ersetzen können.\\
Um so weiter der Testprozess voranschreitet, um so praktischer werden auch die zu erledigenden Aufgaben. Das Potential für eine Automatisierung steigt also im Laufe des Testprozesses.
Fewster und Graham \cite[vgl. S.18]{fewster_software_1999} stellen diesen Zusammenhang in einer Grafik bildlich dar.  Abbildung \ref{fig:intellektuellVsPraktisch} greift diese Darstellung auf und passt sie auf den in Kapitel \ref{sec:testprozess} vorgestellten Testprozess an. Die verschiedenen Möglichkeiten der Testautomatisierung werden in Kapitel \ref{sec:bereiche_der_testautomatisierung} geklärt. Zunächst soll jedoch die Frage beantwortet werden, weshalb eine Automatisierung von Testfällen überhaupt sinnvoll ist.

\begin{figure}[htb]
  \centering  
  \includegraphics[scale=1]{img/intelektuellVsPraktisch.png}\\
  \footnotesize\sffamily\textbf{Quelle:} \cite[vgl. S.18]{fewster_software_1999}
  \caption{Grenzen und Möglichkeiten der Testautomatisierung}
  \label{fig:intellektuellVsPraktisch}
\end{figure}

\section{Warum Testautomatisierung}
\label{sec:warum_testautomatisierung}

Richtig durchgeführt kann Testautomatisierung eine Reihe von Vorteilen bringen. Dustin et al. \cite[S.44 ff.]{dustin_software_2001} stellen drei Hauptvorteile der Testautomatisierung fest:
\begin{itemize}
\item[1.] Erstellung eines zuverlässigen Systems
\item[2.] Verbesserung der Testqualität und Testtiefe
\item[3.] Verringerung des Testaufwands und Reduzierung des Zeitplans
\end{itemize}


In der Literatur gibt es zahlreiche Listen von Vorteilen der Testautomatisierung, die sehr viel feiner gegliedert sind, als die von Dustin et al. \cite[S.44 ff.]{dustin_software_2001} gewählten Oberpunkte.
So nennen beispielsweise Fewster und Graham \cite[vgl. S.9 ff.]{fewster_software_1999} oder auch Thaller \cite[vgl. S.28 ff.]{thaller_software-test_2002} eine Reihe von positiven Aspekten.\\
Gleicht man diese Vorteile mit den von Dustin et al. gewählten Oberpunkten ab, zeigt sich, dass vor allem die Verbesserung der Testqualität und Testtiefe (2), sowie die Verringerung des Testaufwands und die Reduzierung des Zeitplans (3), gut durch diese repräsentiert werden. Diese Oberpunkte lassen sich leicht mit den feiner ausformulierten Vorteilen unterfüttern.\\ Die Erstellung eines zuverlässigen Systems (1) ist hingegen nur schwer direkt durch feiner formulierte Vorteile zu untermauern. In der Regel wird dieser Punkt indirekt durch eine Verbesserung in den letzten beiden Bereichen (2,3) beeinflusst:\\
Eine Verringerung des Testaufwands für einzelne Tests schafft mehr Kapazität, die in bessere und breiter angelegte Tests investiert werden kann. Zusätzlich wird die Testqualität und Testtiefe direkt verbessert. Dies bedingt wiederum, dass mehr Fehler im System aufgedeckt werden können. Dadurch kann eine höhere Qualität des Endproduktes erreicht werden, die sich in einem zuverlässigeren System zeigt.\\
Da die Erstellung eines zuverlässigen Systems (1) nur schwer direkt durch feiner gegliederte Vorteile belegt werden kann und eher eine Folge der Verbesserung in den anderen beiden Bereichen (2,3) ist, wird sie im weiteren nicht näher betrachtet.\\
Fewster und Graham \cite[vgl. S.10]{fewster_software_1999} fassen die Vorteile der Testautomatisierung noch weiter zusammen und reduzieren sich in ihrem Fazit auf die Worte \grq Qualitäts- und Effizienzsteigerung\grq.
Diese Begriffe entsprechen weitestgehend den von Dustin et al. gewählten Oberpunkten. Qualitätssteigerung fasst dabei die Verbesserung der Testqualität und Testtiefe zusammen. Die Verringerung des Testaufwands und Reduzierung des Zeitplans entspricht der Effizienzsteigerung.\\
Um die Vorzüge der Testautomatisierung auf eine feinere und damit greifbarere Ebene zu bringen, werden im Folgenden die Vorteile, wie sie 
Fewster und Graham beschreiben, verwendet und den von Dustin et al. gewählten Oberpunkten zugeordnet.

\subsection{Verringerung des Testaufwands und Reduzierung des Zeitplans}
\label{sec:verringerung_des_testaufwands_und_reduzierung_des_zeitplans}
Die Vorteile in folgender Aufzählung nach Fewster und Graham \cite[vgl. S. 9 ff.]{fewster_software_1999} beschreiben, dass der Aufwand, der für das Testen einer Software betrieben werden muss, mit Hilfe von Automatisierung reduziert werden kann.
Reduzierter Aufwand in den Tests, sowie eine schnellere und wiederholbare Abarbeitung der Testfälle, führen dann meist dazu, dass der gesamte Zeitplan des Projekts positiv beeinflusst wird. Sein volles Potential entfaltet Automatisierung immer dann, wenn Testfälle wiederholt ausgeführt werden. Regressionstests, die vor jedem neuen Releasezyklus einer Software durchgeführt werden, sind daher prädestiniert dazu, automatisiert zu werden. Tester können so von sich wiederholenden Testaufgaben entlastet werden. Das reduziert den Testaufwand und stellt Tester für andere Aufgaben frei, was wiederum das gesamte Projekt beschleunigt.

\begin{itemize}
\item \textit{Ausführen existierender Regressionstests für eine neue Version der Software} \\
Der Aufwand, um Regressionstests manuell durchzuführen, kann schnell sehr groß werden. Sind Testfälle automatisiert, ist es möglich, sie bei Änderungen am System mit wenig Aufwand erneut durchzuführen.
\item \textit{Besserer Einsatz von Resourcen} \\
Mittels Automatisierung lässt es sich vermeiden, Tester durch generischen Aufgaben, wie beispielsweise das immer gleiche erzeugen von Testeingaben, zu binden.
Die frei gewordenen Resourcen können für andere Aufgaben verwendet werden.
Der Zeitplan des Projektes kann so verkürzt werden. 
\item \textit{Wiederverwendbarkeit von Testfällen} \\
Neue Projekte können von den Ergebnissen der Testautomatisierung aus vorangegangenen Projekten profitieren. Auch innerhalb eines Projektes können Teile von automatisierten Testfällen oftmals wiederverwendet werden.
Eine Reduzierung des Zeitplans ist dadurch möglich.
\item \textit{Frühere Markteinführung} \\
Richtig eingesetzt, beschleunigt Testautomatisierung den gesamten Testprozess. Das verkürzt letztendlich auch die Zeit bis zur Markteinführung des Softwareprodukts. 
\end{itemize}


\subsection{Verbesserung der Testqualität und Testtiefe}
\label{sec:verbesserung_der_testqualität_und_testtiefe}
Auch zeigen die beschriebenen Vorteile von Fewster und Graham \cite[vgl. S. 9 ff.]{fewster_software_1999}, dass sich mit Hilfe der Testautomatisierung Verbesserungen im Bereich der Testqualität und Testtiefe erreichen lassen. Eine bessere Testqualität wird meist dadurch erzielt, dass die Testfälle in ihrer Gesamtheit ein höheres Potential erreichen, Fehler aufzudecken. Vor allem eine höhere Testtiefe und eine breitere Testabdeckung sind hier die treibenden Faktoren. Auch die Qualität einzelner Testfälle kann mittels Testautomatisierung direkt verbessert werden. Eine bessere Wiederholbarkeit ist hier der maßgebende Faktor.


\begin{itemize}
\item \textit{Mehr Testfälle öfter ausführen} \\
Aus Zeitmangel müssen sich Tester oft auf einen geringeren Testumfang zurückziehen, als eigentlich gewünscht ist. Vor allem bei sehr generischen Testfällen, die sich beispielsweise nur in verschiedenen Maskeneingaben unterscheiden, ist es mit Hilfe von Testautomatisierung möglich, in weniger Zeit ein Vielfaches an Testfällen durchzuführen.
Eine tiefere Testabdeckung ist die Folge. Da solche Testfälle in der Regel auf einem zentralen Basistestfall beruhen, ist es hier besonders einfach möglich, eine durchgehend hohe Testqualität zu gewährleisten.
\item \textit{Testfälle durchführen, die ohne Automatisierung schwer bis unmöglich wären} \\
Einen Lasttest mit z.B. mehr als 200 Benutzern manuell durchzuführen, erweist sich als nahezu unmöglich. Die Eingaben von 200 Benutzern lassen sich mit Hilfe von automatisierten Lasttests jedoch gut simulieren. Über Testfälle, die ohne Automatisierung gar nicht möglich wären, lässt sich die Testbreite erhöhen. Die Qualität der Tests steigt durch das erhöhte Potential, Fehler zu entdecken.
\item \textit{Konsistenz und Wiederholbarkeit von Testfällen} \\
Testfälle, die automatisch durchgeführt werden, werden immer auf die gleiche Weise ausgeführt. Eine derartige Konsistenz ist auf manuellem Wege kaum zu erreichen. Fehler können somit vermieden und die Qualität gesteigert werden. 
\item \textit{Erhöhtes Vertrauen in die Testfälle und Software } \\
Das Wissen, dass eine Vielzahl an Testfällen erfolgreich vor jedem Release durchgeführt wurden, erhöht das Vertrauen von Entwickler und Nutzern, dass unerwartete Fehler ausbleiben.
Werden Testfälle regelmäßig ausgeführt, erhöht das zusätzlich das Vertrauen, dass diese Testfälle stabil sind und keine Falschmeldungen auftreten.
\end{itemize}


\subsection{Kosten als Bewertungsgrundlage für die Testautomatisierung}
\label{sec:kosten_der_testautomatisierung}
Die unter Kapitel \ref{sec:verbesserung_der_testqualität_und_testtiefe} und Kapitel \ref{sec:verringerung_des_testaufwands_und_reduzierung_des_zeitplans} aufgezeigten Vorteile der Testautomatisierung haben das Problem, dass sie sich meist nur schwer messen und mit genauen Zahlen belegen lassen.
Um das zu erreichen, muss man sich auf die kleinste gemeinsame Größe berufen, auf die all die genannten Vorteile hinarbeiten: Eine Kostenreduzierung beim Testen.
Sowohl eine Verbesserung der Testqualität und Testtiefe, als auch eine Verringerung des Testaufwands und Reduzierung des Zeitplans, verfolgen in letzter Instanz immer das Ziel, Kosten einzusparen.\\
Um den Nutzen der Testautomatisierung für ein Projekt messbar zu machen und nachvollziehbar zu begründen, bieten nach Ramler und Wolfmaier \cite{ramler_economic_2006} daher die direkten Kosten, die durch das Erstellen und Ausführen der Testfälle entstehen, den besten Ansatzpunkt. 
Ramler und Wolfmaier \cite{ramler_economic_2006} zitieren eine Fallstudie von Linz und Daigl \cite{dustin_automated_1999}, die eine Unterteilung der Kosten in zwei Komponenten vornimmt.
\begin{itemize}
    \item[] \(V:=\text{Ausgaben für Testspezifikation und Implementierung}\)
    \item[] \(D:=\text{Ausgaben für einen einzelnen Testlauf}\)
\end{itemize}

Mit Hilfe dieser beiden Variablen können die Kosten (\(A_a\)) für einen einzelnen automatisierten Testfall wie folgt angegeben werden:
\begin{equation}
A_a:=V_a+n*D_a
\end{equation}
\(V_a\) symbolisiert die Kosten, die für die Spezifikation und Implementierung des automatisierten Testfalls anfallen. \(D_a\) die Kosten, die für das einmalige Ausführen des Testfalles entstehen und \(n\) steht für die Anzahl der durchgeführten Testläufe.
Um zu bestimmen, wie sich das manuelle und automatisierte Testen zueinander verhalten, kann analog die selbe Gleichung für das manuelle Testen aufgestellt werden.
\begin{equation}
A_m:=V_m+n*D_m
\end{equation}

Mit Hilfe dieser beiden Gleichungen lässt sich zeigen, dass sich ab einer gewissen Anzahl an Testläufen, die Automatisierung gegenüber der manuellen Ausführung, bezüglich der Kosten lohnt.
Es wird dabei davon ausgegangen, dass die initiale Investition \(V_a\) für die Automatisierung höher ist, als die initiale Investition für das manuelle Testen \(V_m\).
Die Kosten \(A_a\) des automatisierten Testfalls steigen mit jeder Testausführung \(n\) jedoch langsamer an. Beide Funktionen schneiden sich daher in einem Break-Even-Point, ab dem die Automatisierung die günstigere Alternative darstellt.
Abbildung \ref{fig:breakEven} veranschaulicht diesen Zusammenhang grafisch.

\begin{figure}[htb]
  \centering  
  \includegraphics[scale=0.8]{img/breakeven.png}\\
  \footnotesize\sffamily\textbf{Quelle:} \cite{ramler_economic_2006}
  \caption{Break-Even-Point für Testautomatisierung}
  \label{fig:breakEven}
\end{figure}

Zusammenfassend lässt sich feststellen, dass vor allem wiederholt ausgeführte Testtätigkeiten hohes Einsparungspotential bei einer Testautomatisierung bieten.


\subsection{Probleme der Testautomatisierung}
\label{sec:probleme_der_testautomatisierung}
Ramler und Wolfmaier \cite{ramler_economic_2006} zitieren ein Erreichen des Break-Even-Point, wie er in Kapitel \ref{sec:kosten_der_testautomatisierung} beschrieben ist, nach einer Ausführung von 2-20 Testläufen. Diese große Spanne macht deutlich, wie wichtig es ist, in der Testplanung und Steuerung genau abzuwägen, ob und wo eine Automatisierung sinnvoll einzusetzen ist. Eine Automatisierung kann oft auch unwirtschaftlich sein, vor allem immer dann, wenn Tests nur ein einziges mal ausgeführt werden. Implementierungs- und Wartungsaufwand von automatisierten Testfällen sind meist sehr viel höher, als der von manuellen Testfällen. Dieser Mehraufwand muss in irgend einer Weise gerechtfertigt sein. Laut Fewster und Graham \cite[vgl. S. 22 ff.]{fewster_software_1999} und auch Thaller \cite[vgl. S.230 ff.]{thaller_software-test_2002} wird dieser Punkt oftmals vernachlässigt und die Testautomatisierung als Heilmittel für schlecht laufende Prozesse und zu hohe Kosten gesehen. Dabei ist genau das Gegenteil der Fall. Es ist sinnvoller, zunächst die Qualität der Tests und des eigenen Testprozesses zu optimieren, bevor eine Automatisierung eingeführt wird.\\ 
Ein weiterer Schwachpunkt der Automatisierung ist nach Fewster und Graham \cite[vgl. S. 22 ff.]{fewster_software_1999}, dass die Automatisierung von Software-Tests zwar einen Mehraufwand bedeutet, jedoch die Anzahl der gefundenen Fehler nur geringfügig erhöht wird.
Bevor Testfälle automatisiert werden, müssen sie in der Regel zuvor einmal manuell durchgeführt werden, um sicher zu stellen, dass der angedachte Test auch sinnvoll und realisierbar ist. Meist werden Fehler bereits bei dieser manuellen Überprüfung festgestellt. Wird der Testfall dann automatisiert, deckt er weit weniger wahrscheinlich einen Fehler auf, als bei seiner ersten, manuellen Ausführung. Darüber hinaus können automatisierte Testfälle Fehler nur über die Akzeptanzkriterien aufdecken, die ihnen explizit hinterlegt wurden. Oft reichen die hinterlegten Kriterien aber nicht aus, um alle Fehlerquellen abzudecken. Das kann laut Fewster und Graham \cite[vgl. S. 23 ff.]{fewster_software_1999} dazu führen, dass Testfälle als positiv gekennzeichnet werden, obwohl sie in Wirklichkeit fehlerhaft waren. Bei manuellen Tests fallen vergessene Akzeptanzkriterien eher auf bzw. werden durch den Tester intuitiv ergänzt. Die Anzahl der fälschlicherweise positiv gewerteten Testfälle sinkt also mit der manuellen Durchführung.\\
Einen weiteren positiven Aspekt, den Fewster und Graham \cite[vgl. S. 24 ff.]{fewster_software_1999} in einem manuellen Tester sehen, ist eine höhere Stabilität in den Testfällen. Ein Tester kann sich auf kleinere Änderungen in der zu testenden Software leicht einstellen. Er kann selbst entscheiden, ob eine Abweichung vom Testfall als Fehler zu werten oder zu vernachlässigen ist. Automatisierte Testfälle bieten diesen Luxus nicht. Sie verfolgen einen festen Ablaufplan und können nur schwer mit Veränderungen in der zu testenden Software umgehen. Das macht sie, im Vergleich zu manuellen Tests, instabil und erfordert einen erhöhten Wartungsaufwand. In manchen Fällen kann diese Abhängigkeit zwischen Software und automatisiertem Test sogar so weit führen, dass sie die Entwicklung der Software behindern. Beispielsweise dann, wenn eine Änderung so große Auswirkungen auf die automatisierten Testfälle hätte, dass es aus wirtschaftlicher Sicht nicht mehr sinnvoll ist sie durchzuführen. Die Kosten für die Anpassungen wären dann so hoch, dass sie den Nutzen der Softwareänderung übersteigt.\\
Analog zu den in Kapitel \ref{sec:verringerung_des_testaufwands_und_reduzierung_des_zeitplans} und \ref{sec:verbesserung_der_testqualität_und_testtiefe} genannten Vorteilen haben Fewster und Graham \cite[vgl. S. 10 ff.]{fewster_software_1999} auch eine Liste mit bekannten Nachteilen der Testautomatisierung aufgestellt, welche die gerade angesprochenen Probleme noch einmal aufgreifen und ergänzen:

\begin{itemize}
\item \textit{Unrealistische Erwartungen} \\
Testautomatisierung wird oft als Lösung für alle Testprobleme gesehen. Es ist wichtig, dass die Erwartungen aller Beteiligten realistisch bleiben.
\item \textit{Schlechte Testpraxis } \\
Wenn das Testen in einem Unternehmen bereits eine Schwachstelle darstellt, ist es nicht sinnvoll eine Testautomatisierung einzuführen. Es ist besser, zunächst die vorherrschenden Prozesse zu optimieren.
\item \textit{Die Anzahl der gefundenen Fehler wird sich nicht stark verändern } \\
Fehler werden meist bei der ersten Durchführung eines Testfalls aufgedeckt. Wird ein Testfall wiederholt, sinkt auch sein Potential Fehler aufzudecken. In der Regel werden Fehler bereits beim Entwickeln der automatisierten Testfälle entdeckt, nicht erst bei weiteren Testläufen.
\item \textit{Trügerische Sicherheit} \\
Die Tatsache, dass alle automatisierten Testfälle positiv waren bedeutet nicht, dass die Software auch frei von Fehlern ist. Möglicherweise sind nicht alle Bereiche der Software mit Testfällen abgedeckt oder die Akzeptanzkriterien der Testfälle sind nicht umfassend genug gewählt worden. Auch ist es möglich, dass die Testfälle fehlerhaft sind und falsche Ergebnisse anzeigen.
\item \textit{Wartung} \\
 Automatisierte Testfälle haben einen hohen Wartungsaufwand. Änderungen an der Software bedingen oft auch, dass die Testfälle überarbeitet werden müssen. Wird dieser Wartungsaufwand so hoch, dass es günstiger wäre die Testfälle manuell durchzuführen, werden die automatisierten Tests unwirtschaftlich.
\item \textit{Technische Probleme} \\
Die Automatisierung von Testfällen stellt eine komplexe Aufgabe dar und ist daher oft mit Problemen verbunden. Die verwendeten Tools bzw. Frameworks sind meist selbst nicht befreit von Fehlern. Oft gibt es auch technische Probleme mit der zu testenden Software selbst.
\item \textit{Probleme in der Organisation} \\
Eine erfolgreiche Testautomatisierung stellt hohe Anforderungen an die technischen Fähigkeiten der Entwickler und erfordert starken Rückhalt in der Führungsebene. Testautomatisierung läuft nicht immer sofort reibungslos und benötigt oft Anpassungen in vorherrschenden Prozessen.
\end{itemize}


\section{Möglichkeiten der Testautomatisierung im Testprozess}
\label{sec:bereiche_der_testautomatisierung}
Wie bereits zu Beginn dieses Kapitels erwähnt, beschränkt sich die Testautomatisierung nicht nur auf das automatisierte Ausführen von Testfällen, sondern erstreckt sich über den gesamten Testprozess. Innerhalb des Testprozesses haben Amannejad et al. \cite{amannejad_search-based_2014} vier Hauptaufgaben identifiziert, die aus Sicht der Testautomatisierung besonders interessant sind:

\begin{itemize}
\item Testdesign: Erstellen einer Liste von Testfällen.
\item Testcodeerstellung: Erstellen von automatisiertem Testcode.
\item Testdurchführung: Ausführen von Testfällen und Aufzeichnen der Ergebnisse.
\item Testauswertung: Auswerten der Testergebnisse.
\end{itemize}

\subsection{Testdesign}
\label{subsec:testdesign}
Beim Testdesign handelt es sich um eine Aufgabe die laut Thaller \cite[vgl. S. 231]{thaller_software-test_2002} nicht unbedingt prädestiniert dafür ist, automatisiert zu werden. Abbildung \ref{fig:intellektuellVsPraktisch} hat bereits gezeigt, dass es sich um eine eher intellektuell geprägte Aufgabe handelt.
Dennoch gibt es eine Reihe von Möglichkeiten, wie das Entwerfen von Testfällen automatisiert werden kann. Die verschiedenen Ansätze beschränken sich in der Regel darauf, unterschiedliche Testeingaben für die zu testende Software zu finden.
Ein großes Problem, welches die Tools zum Entwerfen der Testfälle laut Fewster und Graham \cite[vgl. S. 19]{fewster_software_1999} dabei haben ist, dass sie einem fest vorgegebenem Algorithmus folgen. Dieses Vorgehen wird jedoch der intellektuellen Komponente dieser Aufgabe nicht gerecht. Ein Tester kann ähnlich strukturierte Testfallerstellungsmethoden heranziehen, wie sie beim automatisierten Testdesign verwendet werden. Um eine komplexe Software umfassend zu testen, reicht es jedoch meist nicht aus, einem festen Algorithmus nachzugehen. Eine tiefere Analyse durch einen Tester wird benötigt. Dieser ist in der Lage, außerhalb eines vorgegebenen Algorithmus, Testfälle zu identifizieren, fehlende Anforderungen zu finden oder sogar, aufgrund von persönlicher Erfahrung, Fehler in der Spezifikation aufzudecken.\\
Ein weiteres Problem das Fewster und Graham \cite[vgl. S. 19]{fewster_software_1999} nennen, ist die Menge an Testfällen, die mit Hilfe von automatisierten Methoden erzeugt werden. Die Anzahl der Testfälle kann schnell so groß werden, dass sie nicht mehr in einem vertretbarem Zeitaufwand durchgeführt werden können. In diesem Fall müssen aus der Menge an Testfällen, die wichtigsten identifiziert werden. Testfälle nach ihrer Relevanz zu filtern ist wiederum eine intellektuelle Aufgabe, die nur schwer in einem Automatisierungstool abgebildet werden kann.\\
Trotz ihrer Probleme haben Methoden zum automatisierten Testfalldesign durchaus ihre Daseinsberechtigung. Sie können die Arbeit des Testers beschleunigen, indem sie ihm einen Grundstock an Testfällen an die Hand geben, der durch weitere Testfälle erweitert werden kann. Sinnvoll ist es, sich nicht ausschließlich auf die Möglichkeit des automatisierten Testfalldesigns zu stützen, sondern sie mit den Fähigkeiten eines Tester zu kombinieren.\\
Seidel et al. \cite[vgl. S. 27]{seidl_basiswissen_2012} beschreiben eine Reihe an Testfallentwurfsmethoden, die sie unter dem Oberbegriff der Kombinatorik zusammenfassen. Diese Entwurfsmethoden zielen darauf ab, aus einer Fülle von möglichen Eingaben diejenigen herauszufiltern, die ein hohes Fehlerpotenzial in sich bergen. Darunter fallen beispielsweise:
\begin{itemize}
\item Äquivalenzklassenbildung
\item Grenzwertanalyse
\item Klassifikationsbaummethode
\end{itemize}
Diese Entwurfsmethoden kommen vor allem beim manuellen Designen von Testfällen zum Einsatz, bieten aber auch die Möglichkeit toolgestützt und damit automatisiert abzulaufen.
Fewster und Graham \cite[vgl. S. 19 ff.]{fewster_software_1999} beschreiben eine Reihe von weiteren Methoden zum Erzeugen von Eingabedaten, die vermehrt in automatisierten Tools zum Einsatz kommen:
\begin{itemize}
\item Codebasierte Generierung von Eingabedaten
\item Interfacebasierte Generierung von Eingabedaten
\item Spezifikationsbasierte Generierung von Eingabedaten
\end{itemize}

\subsubsection{Codebasierte Generierung von Eingabedaten}
\label{subsubsec:codebasierte_generierung}
Die Generierung der Eingabedaten erfolgt bei diesem Ansatz anhand der Struktur des Codes (siehe Abbildung \ref{fig:codeBasedDesign}). Jede Eingabe bedingt einen fest vorbestimmten Ablauf durch das Programm. Anhand des Codes können daher die benötigten Eingaben ermittelt werden, die für das durchlaufen von unterschiedlichen Pfaden im Programm benötigt werden.
Fewster und Graham \cite[vgl. S. 19 ff.]{fewster_software_1999} sehen in diesem Ansatz jedoch Probleme. Ein Testfall benötigt immer auch ein erwartetes Ergebnis. Über die Codebasierte Generierung ist es nicht möglich diese Ergebnisse zu ermitteln. Die generierten Testfälle sind also unvollständig.\\
Ein weiteres Problem dieses Vorgehens ist, dass ausschließlich der Code getestet wird, der bereits implementiert wurde. Fehlende Funktionalitäten können so nicht erkannt werden. Es wird getestet, dass der Code das \grq tut, was er tut und nicht das, was er tun soll.\grq\ \cite[vgl. S. 20]{fewster_software_1999}
\begin{figure}[htb]
  \centering  
  \includegraphics[scale=0.6]{img/codeBasedDesign.png}\\
  \footnotesize\sffamily\textbf{Quelle:} \cite[vgl. S. 19]{fewster_software_1999}
  \caption{Codebasierte Generierung von Testfällen}
  \label{fig:codeBasedDesign}
\end{figure}


\subsubsection{Interfacebasierte Generierung von Eingabedaten}
\label{subsubsec:interfacebasierte_generierung}
Bei dieser Methode erfolgt nach Fewster und Graham \cite[vgl. S. 20]{fewster_software_1999} die Generierung anhand von gut definierten Schnittstellen, wie der Benutzeroberfläche einer Desktop- oder Web-Anwendung (siehe Abbildung \ref{fig:interfaceBasedDesign}). 
Wird als Schnittstelle die Benutzeroberfläche gewählt, kann beispielsweise getestet werden, ob eine Checkbox nach einer Interaktion aktiviert bzw. deaktiviert wurde.\\
Eine weitere Möglichkeit wäre es, rekursiv jeden Link in einer Webanwendung zu durchlaufen. Alle defekten Links der Anwendung könnten auf diese Weise identifiziert werden.\\
Mit Hilfe dieses Ansatzes ist es laut Fewster und Graham \cite[vgl. S. 21]{fewster_software_1999} auch möglich, einfache Akzeptanzkriterien zu generieren. Werden beispielsweise die Links einer Web-Anwendung rekursiv durchlaufen, könnte als Akzeptanzkriterium geprüft werden, ob nach dem ausführen eines Links auch eine neue Webseite geladen wurde.

\begin{figure}[htb]
  \centering  
  \includegraphics[scale=0.6]{img/interfaceBasedDesign.png}\\
  \footnotesize\sffamily\textbf{Quelle:} \cite[vgl. S. 20]{fewster_software_1999}
  \caption{Interfacebasierte Generierung von Testfällen}
  \label{fig:interfaceBasedDesign}
\end{figure}


\subsubsection{Spezifikationsbasierte Generierung von Eingabedaten}
\label{subsubsec:spezifikationsbasierte_generierung}
Mit Hilfe von Spezifikationsbasierter Generierung ist es nach Fewster und Graham \cite[vgl. S. 21]{fewster_software_1999} möglich sowohl Testeingaben, als auch die zugehörigen erwarteten Ergebnisse zu erzeugen (siehe Abbildung \ref{fig:specBasedDesign}). Als Basis wird dazu eine Spezifikation benötigt, die automatisiert analysiert werden kann. Die Möglichkeiten dafür reichen von natürlicher Sprache, die gewissen Strukturen folgt, bis hin zu technischen Modellen.\\ Vor allem die Benutzung von Modellen hat in den letzten Jahren immer mehr an Bedeutung gewonnen und ist heute unter dem Namen \grq modellbasiertes Testen\grq\ bekannt. Als Referenz auf diesem Gebiet kann das Werk von Roßner et al.\ \cite{rossner_basiswissen_2010}, \grq Basiswissen Modellbsierter Test\grq, dienen.\\
Ein Vorteil des Spezifikationsbasierten Ansatzes ist nach Fewster und Graham \cite[vgl. S. 21]{fewster_software_1999} auch, dass die Testfälle nicht auf Basis einer Implementierung, sondern auf Basis einer Spezifikation erzeugt werden. Damit wird sichergestellt, dass sie nicht nur, wie bei der Codebasierten Generierung, überprüfen \grq was die Software tut\grq , sondern \grq was die Software tun soll\grq.

\begin{figure}[htb]
  \centering  
  \includegraphics[scale=0.6]{img/specBasedDesign.png}\\
  \footnotesize\sffamily\textbf{Quelle:} \cite[vgl. S. 21]{fewster_software_1999}
  \caption{Spezifikationsbasierte Generierung von Testfällen}
  \label{fig:specBasedDesign}
\end{figure}

\subsection{Testcodeerstellung}
\label{subsec:testcodeerstellung}
Bei der Testautomatisierung versteht man unter Testcodeerstellung das Erzeugen von Testcode, der später wiederholt gestartet werden kann. Der erzeugte Code setzt die Testfälle um, die zuvor in der Designphase erarbeitet wurden.
In vielen Fällen handelt es sich hierbei um einen manuellen Schritt. Die Testcodeerstellung ist oft eine reine Entwicklertätigkeit, bei der ein Tester die angedachten Testfälle als Testcode implementiert.
Aber auch in diesem Schritt sind laut Amannejad et al. \cite{amannejad_search-based_2014} Möglichkeiten für eine Automatisierung gegeben.\\
Es existieren beispielsweise teilautomatisierte Ansätze. Mit Hilfe von sogenannten \grq record and playback\grq-Tools (R\&PB) können die Interaktionen eines Benutzers mit der zu testenden Software aufgezeichnet werden. Die aufgezeichneten Abläufe können dann verwendet werden, um automatisierte Testskripte zu generieren.\\
Auch ein vollautomatisierter Ansatz ist möglich, wenn auch nicht so weit verbreitet, wie der manuelle bzw. teilautomatisierte Ansatz.
Mittels modellbasiertem Testen ist es nicht nur möglich, wie in Kapitel \ref{subsubsec:spezifikationsbasierte_generierung} angesprochen, Testfalldesigns abzuleiten. Liegen die Modelle in einem entsprechend hohen Detailgrad vor, kann daraus sogar direkt Testcode erzeugt werden. Bouquet et al. \cite{bouquet_test_2008} beschreiben beispielsweise einen modellbasierten Ansatz, mit dessen Hilfe das Designen, Erstellen und Ausführen von Testfällen automatisiert geschehen kann.\\
Zusammengefasst ist die Testcodeerstellung damit in drei unterschiedlichen Automatisierungsgraden möglich:
\begin{itemize}
\item Manuell
\item Teilautomatisiert (R\&PB)
\item Automatisiert
\end{itemize}

Allen Ansätzen gemeinsam ist, dass sie immer eine Schnittstelle zum System benötigen, über welche der automatisierte Testcode mit der zu testenden Anwendung kommunizieren kann.
Meszaros et al. \cite{meszaros_agile_2003} unterscheiden dabei zwei Hauptangriffspunkte:
\begin{itemize}
\item API (application programming interface): Als Schnittstelle wird der Code der zu testenden Anwendung direkt benutzt.
\item UI (user interface): Als Schnittstelle wird die Benutzeroberfläche der Anwendung verwendet.
\end{itemize}

Kombiniert man die verschiedenen Ansätze der Testcodeerstellung mit den Schnittstellen, ergibt sich eine Matrix, welche die verschiedenen Möglichkeiten der Testcodeerstellung beim automatisierten Testen abbildet. Eine grafische Darstellung dieser Matrix bietet Abbildung \ref{fig:bereicheTestcodeerstellung}.


\begin{figure}[htb]
  \centering  
  \includegraphics[scale=0.7]{img/bereicheTestcodeerstellung.png}\\
  \footnotesize\sffamily\textbf{Quelle:} vgl. \cite{meszaros_agile_2003}
  \caption{Verschiedene Möglichkeiten der Testcodeerstellung}
  \label{fig:bereicheTestcodeerstellung}
\end{figure}

Meszaros et al. \cite{meszaros_agile_2003} haben eine ähnliche Matrix aufgestellt, die noch um eine weitere Dimension der Testgranularität (Unit-, Integrations- und System-Test) erweitert ist. Diese Dimension hat allerdings nur wenig Auswirkung auf die Herangehensweise in der Testautomatisierung und wurde deshalb in Abbildung \ref{fig:bereicheTestcodeerstellung} nicht mit aufgeführt.
Im Folgenden wird, mit Hilfe von Beispielen, auf die Schnittstellen API und UI in Kombination mit der Manuellen- und R\&PB-Testcodeerstellung näher eingegangen.
Die vollautomatisierte Testcodeerstellung mittels modellbasierten Testen, dargestellt durch die beiden rechten Quader in Abbildung \ref{fig:bereicheTestcodeerstellung}, befindet sich außerhalb des Rahmens dieser Arbeit und wird daher nicht genauer betrachtet.

\subsubsection{API}
\label{subsubsec:API}

Unter der Abkürzung API sind in diesem Zusammenhang alle Schnittstellen zu verstehen, die intern von der zu testenden Anwendung angeboten werden. Darunter fällt beispielsweise das direkte aufrufen von Servicemethoden, die als Businesslogik innerhalb der Anwendung bereitgestellt werden.\\
Eine weit verbreitete Gruppe von Frameworks, welche oft diese Art der Schnittstelle verwenden, sind die modernen \grq XUnit-Frameworks\grq. In den meisten Programmiersprachen existiert ein Unit-Framework, mit dem es möglich ist, direkt die im Programm angebotene Logik mit automatisierten Testfällen zu überprüfen. In der Praxis wird diese Aufgabe meist direkt von einem Entwickler, in Form von Unit-Tests (Komponenten-Tests) übernommen. Der Begriff \grq XUnit-Frameworks\grq\ kann leicht missverstanden werden, da er impliziert, dass es sich bei den mit Hilfe des Frameworks umgesetzten Testfällen immer um Unit-Tests handelt. Es können jedoch Testfälle aller Teststufen, von Unit- bis System-Test, mit Hilfe dieser Frameworks umgesetzt werden.
Die Testcodeerstellung erfolgt dabei manuell, womit diese Art der Automatisierung ein Beispiel für den oberen linken Quader in Abbildung \ref{fig:bereicheTestcodeerstellung} darstellt.\\
Neben dem manuellen Vorgehen beschreiben Meszaros et al. \cite{meszaros_agile_2003} auch die Möglichkeit \grq record and playback\grq\ für Testfälle zu verwenden, die als Schnittstelle die API ansteuern.
Hierfür muss eine \grq record and playback\grq-Mechanismus auf API-Ebene für die zu testende Software entwickelt werden. Darüber ist es dann bei einem manuellem Testlauf möglich, all das aufzuzeichnen, was den Zustand des Systems beeinflusst hat. Aus diesen Aufzeichnungen kann dann Testcode erzeugt werden, der es ermöglicht die gewünschten Abläufe zu wiederholen. Solch ein Vorgehen wäre ein Beispiel für den oberen mittleren Quader in Abbildung \ref{fig:bereicheTestcodeerstellung}. Diese Art der Testcodeerstellung findet aber keine verbreitete Anwendung.

\subsubsection{UI}
\label{subsubsec:UI}
Bei Testfällen, die als Schnittstelle zum System das User-Interface verwenden, ist laut  Meszaros et al. \cite{meszaros_agile_2003} vor allem \grq record and playback\grq\ ein sehr weit verbreiteter Ansatz. Es existieren zahlreiche kommerzielle Testwerkzeuge, um dieses Vorgehen zu unterstützen, wie beispielsweise HP Unified Functional Testing \cite{hp_testautomatisierung_2015}, aber auch Open-Source-Lösungen wie Selenium \cite{selenium_selenium_2015}.\\
Diese Tools ermöglichen es Abläufe aufzuzeichnen, um sie später automatisch zu wiederholen. Die Testfälle werden dafür zunächst manuell auf der Benutzeroberfläche der Anwendung durchgeführt. Während dieser manuellen Testausführung können die vom Tester getätigten Eingaben vom Tool gespeichert werden. Die gespeicherten Interaktionen werden dann verwendet, um daraus Testcode zu generieren, mit dem die Abläufe auf der Benutzeroberfläche beliebig oft wiederholt werden können. Dieses Art der Testcodeerstellung wird in Abbildung \ref{fig:bereicheTestcodeerstellung}, durch den unteren mittleren Quader dargestellt.\\
Der untere linke Quader steht für manuell erstellten Testcode, der als Schnittstelle zum System das User-Interface verwendet. Oft handelt es sich dabei um Testfälle, die analog zu Abschnitt \ref{subsubsec:API} ein \grq XUnit-Frameworks\grq\ zur Testausführung verwenden. Diese Frameworks können um Aufsätze, wie beispielsweise HttpUnit \cite{httpunit_httpunit_2015} oder Selenium \cite{selenium_selenium_2015} erweitert werden. Mit Hilfe dieser Erweiterungen ist es möglich, Testfälle nicht mehr nur gegen die API der zu testenden Software zu entwickeln, sondern zur Steuerung die Benutzeroberfläche zu verwenden. Neben Aufsätzen für vorhandene Frameworks existieren auch zahlreiche eigenständige Tools, wie beispielsweise MicroFocus SilkTest \cite{silk_test_borland_2015}, welche neben Webanwendungen auch die Möglichkeit zum Testen von Desktopanwendungen bieten.\\
Testcode von Hand zu schreiben ist ein weit verbreitetes Vorgehen. Vor allem immer dann, wenn kein vorhandenes Testtool verwendet wird, sondern das Automatisierungsframework für die zu testende Software selbst entwickelt wurde.


\subsection{Testdurchführung}
\label{subsec:testdurchführung}
Unter Testdurchführung versteht man das Ausführen der Testfälle, sowie das Aufzeichnen von Testergebnissen bzw. Testausgaben. Nach den Erfahrungen von Amannejad et al. \cite{amannejad_search-based_2014} ist dieser Bereich der Testautomatisierung derjenige, der von den meisten Testern am engsten mit der Automatisierung verbunden wird.\\
Die Möglichkeit zur Automatisierung hängt in diesem Schritt stark von den gewählten Methoden in den vorangegangenen Phasen des Testprozesses ab. Oft ist eine automatisierte Testdurchführung bereits ohne zusätzlichen Aufwand möglich. Vor allem dann, wenn der Testcode zuvor Tool bzw. Framework unterstützt erstellt wurde.
Die meisten Tools und Frameworks ermöglichen bereits eine automatisierte Testdurchführung.\\
Im besonderen Maße sind an dieser Stelle, aufgrund ihrer hohen Verbreitung, erneut die \grq XUnit-Frameworks\grq\ zu nennen.
In Abschnitt \ref{subsec:testcodeerstellung} wurde aufgezeigt, dass diese Frameworks oft in der manuellen Testcodeerstellung verwendet werden. Das Unit-Framework übernimmt dabei die automatisierte Ausführung des Testcodes, sowie die Präsentation der Testergebnisse und ist damit ein Beispiel für ein Framework, welches die Testdurchführung automatisiert.\\
Probleme ergeben sich, wenn Testcode losgelöst von bereits vorhandenen Tools bzw. Frameworks entwickelt wurde. Der Testcode könnte beispielsweise gekapselt in Funktionen vorliegen und seine Ergebnisse in die Konsole loggen. Steigt die Anzahl der Testfälle, wird die Ausführung und vor allem die folgende Auswertung zum Problem. Der Aufwand, der später betrieben werden muss, um die Testfälle anhand ihrer Logmeldungen auszuwerten, kann so eine Dimension erreichen, die nicht mehr wirtschaftlich ist. Ein weiteres Problem stellt dar, dass eine gesonderte Ausführung von einzelnen Testfällen bzw. kleineren Testfallgruppen nur mit zusätzlichem Aufwand möglich ist.\\
Werden Testfälle daher ohne Tool- bzw. Framework-unterstützung erstellt, müssen zeitnah Überlegungen angestellt werden, wie diese einfach und auswertbar zur Ausführung gebracht werden können.
Diese Überlegungen resultieren meist in einem eigen entwickelten Testframework, welches die Ausführung des automatisierten Testcodes und die Aufzeichnung der Ergebnisse übernimmt.


\subsection{Testauswertung}
\label{subsec:testauswertung}
Nachdem die zu testende Software mittels eines Testfalls ausgeführt wurde, muss bestimmt werden, ob dieser erfolgreich oder fehlerhaft war. Diese Prüfung kann manuell erfolgen. Handelt es sich jedoch um automatisierte Testfälle, ist diese Prüfung meist über feste Akzeptanzkriterien direkt im Testfall hinterlegt. Im Fall von Testfällen, deren Durchführung über ein \grq XUnit-Frameworks\grq\ automatisiert wurde, werden Akzeptanzkriterien beispielsweise über sogenannte \grq Assertions\grq\ festgelegt.\\
Um die erwarteten Ergebnisse für einen Testfall zu finden, werden sogenannte Testorakel verwendet.
Jede Quelle, die Auskunft über ein zu erwartendes Ergebnis gibt, kann als Testorakel dienen.
Meist handelt es sich um Spezifikationen, die manuell ausgewertet werden.
Darüber hinaus gibt es aber auch zahlreiche Möglichkeiten und Versuche, diese zu automatisieren \cite{memon_automated_2000} \cite{richardson_specification-based_1992} \cite{shahamiri_comparative_2009}.
Automatisierte Testorakel können dann wiederum für die Auswertung der Testergebnisse dienen. Eine manuelle Auswertung kann so entfallen.\\
Bereits mit fest hinterlegten Akzeptanzkriterien, die manuell gefunden wurden, gilt laut Amannejad et al. \cite{amannejad_search-based_2014} die Testauswertung als automatisiert. Mittels automatisierter Testorakel kann diese jedoch noch ein höheres Level an Intelligenz erreichen. 





\newpage
\chapter{Testautomatisierung mit Selenium}
\label{sec:testautomatisierung_mit_selenium}

Laut Seidel et al. \cite[vgl. S. 48]{seidl_basiswissen_2012} ist der am häufigsten genutzte \frqq Angriffspunkt\flqq\ für Testautomatisierung die grafische Benutzerschnittstelle. Seidel et al. \cite[S. 48]{seidl_basiswissen_2012} nennen dafür folgende Gründe:
\begin{itemize}
\item \glqq Sie ist für Tester und Automatisierer anschaulich und leicht greifbar.\grqq
\item \glqq Sie stellt zumeist das Verhalten im realen Umfeld am besten nach.\grqq
\item \glqq Die Dokumentation von Systemen ist auf dieser Ebene meist am vollständigsten.\grqq
\item \glqq Der klassische Systemtest wird oft über diese Schnittstelle abgewickelt.\grqq
\item \glqq Hinter der grafischen Benutzerschnittstelle liegende Systeme werden implizit getestet.\grqq
\end{itemize}
Ein großer Teil der heutzutage entwickelten Anwendungen werden in Form einer Webapplikation realisiert. Für automatisierte Testfälle, die als Schnittstelle die grafische Benutzeroberfläche verwenden, stellen diese Webapplikation einen Sonderfall dar. Elemente auf der Oberfläche können besonders gut angesprochen werden.
Im Gegensatz zu gewöhnlichen Denktopanwendungen gibt es bei dieser Art von Anwendung, wie Seidel et al. \cite[vgl. S. 88]{seidl_basiswissen_2012} feststellen, \glqq keinen spezifischen Client für eine Applikation, sondern einen generischen - den Browser.\grqq\ Dies schafft nach Seidel et al.  \cite[vgl. S. 59]{seidl_basiswissen_2012} \glqq für Werkzeuge eine sehr gute Basis, um auf die Elemente der Seite zuzugreigen.\grqq\ Die einzelnen HTML-Elemente und deren Attribute können verwendet werden um die Bestandteile eine Seite zu adressieren.
Ein weit verbreitetes Tool für die Automatisierung, welches diesen Ansatz verfolgt ist Selenium \cite{selenium_selenium_2015}.
\section{Selenium}
\label{sec:selenium}
Seidel et al. \cite[S. 142]{seidl_basiswissen_2012} beschreiben Selenium als \glqq eines der gängigsten Open-Source-Automatisierungswerkzeuge für Webapplikationen.\grqq\\
Ordnet man Selenium der in Kapitel \ref{subsec:testcodeerstellung} gewählt Unterteilung in der Testcodeerstellung zu, ist es im unteren mittleren und unteren linken Quadranten angesiedelt. Abbildung \ref{fig:bereicheTestcodeerstellungSelenium} hinterlegt diese Bereiche farblich.
Selenium ist also ein Tool mit welchem Testfälle erstellt werden können, die als Schnittstelle die grafische Benutzeroberfläche eine Webanwendung verwenden. Die Testcodeerstellung erfolgt dabei manuell oder teilautomatisiert mittels \grq record-and playback\grq.\\

\begin{figure}[htb]
  \centering  
  \includegraphics[scale=0.7]{img/bereicheTestcodeerstellungSelenium.png}\\
  \footnotesize\sffamily\textbf{Quelle:} vgl. \cite{meszaros_agile_2003}
  \caption{Einordnung von Selenium in die verschiedene Möglichkeiten der Testcodeerstellung}
  \label{fig:bereicheTestcodeerstellungSelenium}
\end{figure}

Genau genommen handelt es sich bei Selenium aber nicht um ein einzelnes Tool, sondern um eine Reihe von Tools, die unter dem Namen Selenium zusammengefasst werden.
In seiner aktuellen Ausprägung 2.x lassen sich, abgesehen von Komponenten die der Abwärtskompatibilität dienen, laut Dokumentation \cite{selenium_selenium_2015-1}, drei Komponenten unterscheiden:

\begin{itemize}
\item Selenium IDE \\
Bei der Selenium IDE handelt es sich um ein Firefox plug-in, das verwendet werden kann, um Selenium-Testscripte zu erstellen. Testscripte können von Hand erstellt oder mittels eines \grq record-and playback\grq -Mechanismus direkt im Browser aufgezeichnet werden. Die erstellten Testfälle können mit Akzeptanzkriterien angereichert und innerhalb der IDE wieder abgespielt werden.
\item Selenium WebDriver \\
Der Selenium WebDriver bietet für verschiedene Programmiersprachen eine API zur Steuerung eines Browsers aus dem Programmcode heraus. Der WebDriver bildet damit die Kernkomponente für alle Selenium-Testfälle die außerhalb der Selenium IDE entwickelt werden.

\item Selenium Server/Grid \\
Mit Hilfe des Selenium Servers ist es möglich Selenium-Testfälle nicht nur auf dem eigenen Rechner auszuführen, sondern die Ausführung auf einen Server auszulagern. Einen wichtigen Teil des Selenium Server bildet Selenium Grid. Selenium Grid bietet die Möglichkeit, die Ausführung von Selenium-Testfällen über einen Server hinaus auf eine Vielzahl von Knoten zu verteilen. Selenium Server dient dann als Hub, der die Testfallanfragen auf registriert Knoten zur Ausführung weiterleitet. 
\end{itemize}


\section{Testdurchführung mit Selenium}
\label{sec:testdurchführung_mit_selenium}
Abhängig davon, ob die Testfälle für die Selenium IDE entwickelt wurden oder sich auf den Selenium WebDriver stützen, unterscheiden sich die Möglichkeiten zur Ausführung der Testskripte.\\
Testfälle die mit der Selenium IDE entwickelt wurden, verwenden eine Selenium eigene Sprache mit dem Namen \grq Selense\grq.\ Diese Testfälle können in späteren Testläufen wieder über die Selenium IDE zur Ausführung gebracht werden. \\
Dem gegenüber stehen Testfälle, die den Selenium WebDriver verwenden.
Testfälle die mittels Selenium WebDriver entwickelt wurden sind für ihre Ausführung nicht an ein spezielles Tool gebunden. Beim WebDriver handelt es sich um eine API, mit deren Hilfe ein Browser ferngesteuert werden kann. Wie diese API in die Testfälle integriert wird, ist dem Entwickler überlassen. In der Praxis hat sich als Best Practice herausgestellt, den WebDriver verwenden in Verbindung mit einem Unit-Framework einzusetzen.
Im Java-Umfeld wären hier beispielsweise JUnit oder TestNG zu nennen.
Die Testfälle können dann analog zu den klassischen Unit Tests entwickelt werden, verwenden jedoch als Schnittstelle zum System nicht die API sondern, über den WebDriver, die Benutzeroberfläche der Webanwendung. 
Die Ausführung der Testfälle erfolgt dann identisch zu den klassischen Unit Tests.\\
Bei Verwendung des WebDrivers stützt sich Selenium damit auf bereits sehr gut etablierte Frameworks. Das hat den Vorteil, dass die so erstellten Testfälle, bereits in den meisten Programmiersprachen gut in die Infrastruktur integriert sind. In Java sind Testfälle die mittels JUnit ausgeführt werden in allen gängigen IDEs durch Plugins unterstützt. Noch viel wichtiger ist jedoch eine gute Integration der Testfälle in den Buildprozess. Werden in Java Standarttools, wie beispielsweise Gradle oder Maven zum bauen der Projekte verwendet, können die Testfälle ohne Mehraufwand im Rahmen des Buildprozesses ausgeführt werden.



\section{Testcodeerstellung mit Selenium}
\label{sec:Testdesign}
Die Testcodeerstellung von Testfällen, die Selenium verwenden, kann auf zwei Arten erfolgen.
Testcode kann manuell erstellt werden oder über einen \grq record-and playback\grq -Mechanismus teilautomatisiert generiert werden.

\subsection{Recorde-and-playback}
\label{sec:recorde_and_playback}
Die Selenium IDE bietet die Möglichkeit Testskripte mittels \grq record and playback\grq\ zu erzeugen.
Die im Testfall gewünschten Abläufe werden dazu einfach im Browser abgearbeitet.
Die elenium IDE zeichnet während dessen die durchgeführten Schritte in der Selenium eigenen Sprache \grq Selense\grq\ auf. Diese Aufzeichnungen können später wieder von der Selenium IDE interpretiert werden, um so die Testabläufe erneut zu wiederholt.
Testskripte können nach dem Aufzeichnen überarbeitet werden. Auf diese Weise ist es beispielsweise möglich, Akzeptanzkriterien an bestimmten Stellen im Testablauf einzuarbeiten.\\
Testfälle die über den \grq record-and playback\grq -Mechanismus erstellt wurden, sind nicht an die Sprache \grq Selense\grq\ gebunden. Die Selenium IDE bietet die Möglichkeit, Testskripte in eine Reihe von Programmiersprachen zu exportieren, darunter auch Java.
Diese Testfälle benutzen dann, wie in Abschnitt \ref{sec:testdurchführung_mit_selenium} beschrieben, den Selenium Webdriver für die Kommunikation mit dem Browser und ein Unit-Framework für die Ausführung.\\
Die \grq record and playback\grq -Funktionalität bietet einen besonders einfachen und schnellen Weg, um Testfälle zu erstellen. Dennoch wird in der Dokumentation von Selenium \cite{selenium_selenium_2015-1} davon abgeraten, sich bei der Testfallerstellung alleine auf dieses Tool zu stützen. Die Selenium IDE wird als Prototyping-Tool verstanden, mit dem kleine Aufgaben, die nicht für den längerfristigen Einsatz gedacht sind, schnell automatisiert werden können.

\subsubsection{Probleme von recorde-and-playback}
\label{sec:probleme_von_recorde_and_playback}
Testabläufe die über den \grq record-and playback\grq -Mechanismus in der IDE erstellt werden unterliegen eine Reihe von Limitierungen. Leotta et al. \cite{leotta_repairing_2013} nennen als Limitierungen dieser Testfälle, das Fehlen von bedingten Anweisungen, Schleifen, Logging, Ausnahmebehandlungen so wie parametrisierten (a.k.a. data-driven) Testfällen.
Neben diesen Limitierungen haben die Testfälle zusätzlich das Problem, dass sie eine schlechte Wartbarkeit aufweisen. Nach Leotta et al. \cite{leotta_repairing_2013} liegt das vor allem daran, dass die Testfälle sehr stark mit der Struktur der Webseiten verwoben sind und einen hohen Anteil an dupliziertem Code aufweisen.
Die Limitierungen der IDE können zwar durch das Exportieren der Testfälle in eine Programmiersprache überwunden werden, die Qualität der Testfälle im Bezug auf ihre spätere Wartbarkeit kann nach  Leotta et al. \cite{leotta_repairing_2013} auf diesem Wege jedoch nicht verbessert werden.\\
Um die starke Koppelung mit der Webanwendung und den hohen Anteil an dupliziertem Code zu veranschaulichen, wurden zwei Testfälle über den \grq record-and playback\grq -Mechanismus von Selenium erstellt und in die Programmiersprache Java exportiert.
Die beiden Testfälle sollen das Anlegen und Editieren eines Datensatzes auf einer einfachen Webanwendung überprüfen.
Drei Seiten der Anwendung werden für diesen Testfall verwendet. Die Seiten sind in Abbildung \ref{fig:toDoApp} dargestellt.

\begin{figure}[htb]

 \subfigure[Anlegen eines neuen Datensatzes]{\includegraphics[width=0.49\textwidth]{img/newTodo.png}}
  \subfigure[Anzeigen eines Datensatzes]{\includegraphics[width=0.49\textwidth]{img/showTodo.png}}
 \subfigure[Editieren eines Datensatzes]{\includegraphics[width=0.49\textwidth]{img/editTodo.png}}
 \caption{Anlegen, editieren und anzeigen eines neuen Datensatzes}
  \label{fig:toDoApp}
\end{figure}


Zur besseren Lesbarkeit wurden die Testfälle im Listing \ref{lst:exportierteTestfaelle} leicht überarbeitet, in ihrem Wesen jedoch nicht verändert.\\
\begin{lstlisting}[caption={Exportierte Testfälle},label={lst:exportierteTestfaelle}]
  /**
  * Testfall legt einen neuen Datensatz an.
  */
  @Test
  public void testCreateNewRecord() {
    WebDriver driver = new FirefoxDriver();
    driver.get("http://localhost:3000/todos/new");
    driver.findElement(By.id("todo_title")).sendKeys("MyTitle");
    driver.findElement(By.id("todo_notes")).sendKeys("MyNote");
    driver.findElement(By.name("commit")).click();
  
    assertEquals("MyTitle", driver.findElement(By.id("title")).getText());
    assertEquals("MyNote", driver.findElement(By.id("note")).getText());
  }
  
  /**
  * Testfall legt einen neuen Datensatz an und editiert ihn.
  */
  @Test
  public void testEditRecord() {
    WebDriver driver = new FirefoxDriver();
    driver.get("http://localhost:3000/todos/new");
    driver.findElement(By.id("todo_title")).sendKeys("MyTitle");
    driver.findElement(By.id("todo_notes")).sendKeys("MyNote");
    driver.findElement(By.name("commit")).click();
    driver.findElement(By.linkText("Edit")).click();
    driver.findElement(By.id("todo_title")).clear();
    driver.findElement(By.id("todo_title")).sendKeys("MyTitleEdit");
    driver.findElement(By.id("todo_notes")).clear();
    driver.findElement(By.id("todo_notes")).sendKeys("MyNoteEdit");
    driver.findElement(By.name("commit")).click();
    
    assertEquals("MyTitleEdit", driver.findElement(By.id("title")).getText());
    assertEquals("MyNoteEdit", driver.findElement(By.id("note")).getText());
  }
  
\end{lstlisting}
Die Testfälle für das Anlegen und Editieren sind in ihren Abläufen recht ähnlich. Beide Testfälle legen zunächst einen neuen Datensatz in der Anwendung an. Die Logik, die hierfür verwendet wird, ist in beiden Testfällen dupliziert worden. Das duplizieren von Code ist nicht nur ein Problem dieses Beispiels sondern ein Problem, dass durchaus hohe Praxisrelevanz hat. In den meisten Testabläufen finden sich wiederkehrende Aufgaben, wie beispielsweise einen Login, der nicht nur von einem Testfall benötigt wird.
Selbst wenn sich die Abläufe zwischen den Testfällen stark unterscheiden werden Elemente, beispielsweise Input-Felder, die von der Anwendung angeboten werden, in der Regel nicht nur einmal benutzt. Das führt zwangsläufig dazu, dass die Logik zum adressieren dieser Felder in einer Vielzahl von Testfällen verwendet wird. Ein hohe Anzahl an dupliziertem Code ist die Folge.\\
Die Adressierung der Elemente in der Anwendung bedingt darüber hinaus auch die hohe Koppelung der Testfälle mit den Seiten der Anwendung, welche Leotta et al. \cite{leotta_repairing_2013} als weiteres Problem identifiziert haben. Die Selektoren die verwendet werden um die Elemente der Webanwendung anzusteuern sind direkt in den Testfall eingearbeitet. Änderungen an der Seitenstruktur der Anwendung haben damit direkten Einfluss auf die Testfälle.
Obwohl sich die eigentliche Testfallspezifikation durch eine Änderung in der Seitenstruktur nicht ändert, müssen die Testfälle daher trotzdem überarbeitet werden.\\
Der duplizierter Code und der hohe Grad an Koppelung mit der Anwendung innerhalb der Testfälle bedingen, dass selbst kleine Änderungen in der zu testenden Anwendung, Korrekturen an vielen Stellen in den Testfällen nötig machen.

\subsection{Manuell}
\label{sec:manuell}
Eine weitere Möglichkeit bildet das manuelle Erstellen der Testskripte. Für die Ausführung und die Kommunikation mit der Anwendung verwenden die manuell erstellten Skripte das selbe Toolset, wie die mittels \grq record-and playback\grq generierten und exportierten Testfälle. Analog zu den in Listing \ref{lst:exportierteTestfaelle} gezeigten Testfällen wird auch bei diesen Skripten der Selenium WebDriver für die Kommunikation mit der Anwendung verwendet. Die Ausführung geschieht in der Regel ebenfalls über ein Unit-Fraemwork.\\
Im Vergleich zu den generierten Testfällen ist das manuelle entwickeln der Testskripte aufwändiger.
Es bietet jedoch die Möglichkeit die in Abschnitt \ref{sec:probleme_von_recorde_and_playback} genannten Problemen der \grq record-and playback\grq -Variante entgegenzuwirken.\\
Bei einem manuellen Ansatz kann von Anfang an auf eine wartbare und wiederverwertbare Struktur in den Testfällen geachtet werden.
Als best practice hat sich zu diesem Zwecke das Page Object Design Pattern durchgesetzt.

\subsection{Page Object Pattern}
\label{sec:page_object_pattern}
Im Page Object Pattern wird versucht, die Funktionalität, welche die zu testende Anwendung anbietet, in einem objektorientierten Ansatz zu kapseln.
Alle Seiten der zu testenden Anwendung werden dazu als Kassen, so genannten Page Objects, modelliert. Jede dieser Klassen verwaltet zentral alle Informationen so wie die Funktionalität die von der jeweils korrespondierende Webseite angeboten wird.\\
Ein Page Objekt ist also eine objektorientierte Klasse die als Interface für eine Seite der zu testenden Anwendung dient.
Sämtliche Interaktion mit der zu testenden Anwendung geschieht über die in den Page Objekts angeboten Schnittstellen.
Änderungen an der Oberfläche der zu testenden Anwendung haben so keinen direkten Einfluss mehr auf die Testfälle. Bei Änderungen an der Benutzeroberfläche muss nur noch Code an einer Stelle, innerhalb der Page Objekts, angepasst werden.\\
Um das Zusammenspiel zwischen Page Objekts und Testfall besser zu verdeutlichen wurde der Test zum anlegen eines neuen Eintrags aus Listing \ref{lst:exportierteTestfaelle} mit dem Page Object Pattern nachgebaut.\\
Der Testfall Arbeitet mit zwei Seiten der Anwendung. Die Seiten wurden bereits in Abbildung \ref{fig:toDoApp} dargestellt. Für jede der beiden Seiten wird ein Page Object als Kommunikationsschnittstelle benötigt. 
Das Page Object CreatPage in Listing \ref{lst:poCreatePage} repräsentiert die Seite zum anlegen eines neuen Datensatzes (Abbildung \ref{fig:toDoApp} a). Das Page Object ShowPage in Listing \ref{lst:poShowPage} repräsentiert die Seite zum anzeigen eines Datensatzes (Abbildung \ref{fig:toDoApp} b).

\begin{lstlisting}[caption={Page Object CreatePage},label={lst:poCreatePage}]
  public class CreatePage extends BasePo {
		public final Control tfTodotitle = control(by.textField("todo_title"));
		public final Control tfTodonotes = control(by.textField("todo_notes"));
		public final Control bCreateTodo = control(by.button("commit"));
		
		public CreatePag(PageObject po) {
			super(po);
		}
		public ShowPage createEntry(String title, String note){
			tfTodotitle.sendKeys(title);
			tfTodonotes.sendKeys(note);
			bCreateTodo.click();
			return new ShowPage(this);
		}
  }
\end{lstlisting}  
\begin{lstlisting}[caption={Page Object ShowPage},label={lst:poShowPage}]  
  public class ShowPage extends BasePo {
		public final Control idTitle = control(by.id("title"));
		public final Control idNotes = control(by.id("note"));

		public ShowPage(PageObject po) {
			super(po);
		}
  }
\end{lstlisting}

Die Funktionalitäten die von den jeweiligen Seiten angeboten werden sind innerhalb des Page Objects gekapselt. Das Page Object CreatPage bietet beispielsweise die Eingabefelder für Titel und Note, so wie den Button zum anlegen eines Datensatzes, als globale Objekte von Typ Control an.\\
Die Klasse Control dient in den dargestellten Page Objekts als Wrapper für den Selenium WebDriver. Control-Objekte sind also analog zu Selenium WebElementen zu verstehen.\\
Die Funktionalität zum anlegen eines neuen Eintrags, die im späteren Testfall benötigt wird, bietet das Page Objekt CreatPage als Methode \grq createEnty()\grq\ an.\\

Der Testfall in Listing \ref{lst:pageObjectTestfall} verwendet beiden Page Objekts um einen neuen Eintrag in der Anwendung anzulegen und zu überprüfen.


\begin{lstlisting}[caption={Page Object Testfall},label={lst:pageObjectTestfall}]  
  /**
  * Testfall legt einen neuen Datensatz an.
  */
 	@Test
	public void testCreateNewRecord(){
		CreatePage createPage = new CreatePage(po);
		ShowPage showPage = createPage.createEntry("MyTitle", "MyNote");

		assertEquals("MyTitle", showPage.idTitle.resolve().getText());
		assertEquals("MyNote", showPage.idNotes.resolve().getText());
	}

\end{lstlisting}

Im Gegensatz zum Testfall in Listing \ref{lst:exportierteTestfaelle} ist es auf Grund des Page Object Pattern nicht mehr nötig explizite Referenzen auf die Struktur der Seite innerhalb der Testfälle zu machen. Alle Details der Seite sind innerhalb der Page Objekts gekapselt. Der Testfall verwendet lediglich die im Page Object angebotene Funktionalität.


\subsubsection{Vorteile des Page Object Pattern}
\label{sec:vorteile_des_page_object_pattern}

Folgende Vorteile ergeben sich damit bei der Verwendung des Page Object Pattern die auch so in der Dokumentation von Selenium \cite{selenium_selenium_2015-2} angegeben werden:

\begin{enumerate}
\item Es gibt eine klare Trennung zwischen Testcode und seitenspezifischem Code, wie beispielsweise Elementadressierungen und Layout. \\ \\
Die Adressierung der Elemente ist nicht mehr über die gesamten Testfälle verteilt sondern befindet sich an einer zentralen Ort, dem Page Object.
Die hohe Koppelung der Testfälle mit den Seiten der Anwendung, die Leotta et al. \cite{leotta_repairing_2013} als Problem genannt haben, kann somit überwunden werden.

\item Es gibt einen einzigen Ort für die Elemente und Operationen die von einer Seite angeboten werden. \\ \\
Alle Informationen die eine Seite der Anwendung betreffen sind an einem zentralem Ort, dem Page Object gesammelt. Seitenspezifischer Code muss somit nicht mehr in den einzelnen Testfällen dupliziert werden. Die Funktionalitäten der Seite können über das entsprechende Page Object abgerufen werden.

\end{enumerate}

Beide Vorteile führen dazu, dass Änderungen die an einer Seite gemacht werden nur Auswirkungen an einem zentralen Stelle haben. Die Wartbarkeit der gesamten Testfälle wird so erhöht.
Leotta et al. unterstützen diese These mit einer Fallstudie \cite{leotta_repairing_2013} in der Sie eine herkömmliche Testsuite mit einer Testsuite, die das Page Object Pattern implementiert, hinsichtlich ihrer Wartbarkeit verglichen haben.
Das Ergebnis dieser Studie hat gezeigt, dass in Ihrem Fall, die Zeit für die Wartung der Testfälle um ca. 65\% reduziert werden konnten. Die Anzahl der anzupassenden Codezeilen konnte um ca. 87\% reduziert werden.

\subsubsection{Probleme des Page Object Pattern}
\label{sec:probleme_des_page_object_pattern}

Die Vorteile des Page Object Pattern kommen allerdings mit einem Preis. Die Komplexität des gesamten Testprojekts steigt. Testfälle können nicht mehr beliebig programmiert werden sondern sind in den Kontext von Page Objekts zu stellen. Das erfordert tiefgründige Designentscheidungen. So ist beispielsweise zu klären, wie über die Page Objekts ein generischer Einstieg in die Anwendung angeboten werden kann oder wie der WebDriver über mehrere Page Objekts und Testfälle hinweg verwaltet wird. Darüber hinaus müssen die Page Objects zunächst initial entwickelt werden, bevor mit dem erstellen von Testfällen begonnen werden kann.\\
Verglichen mit einem herkömmlichen Ansatz steigt mit der Verwendung des Page Object Pattern also zu Beginn der Aufwand beim erstellen eines Testprojektes.\\
Wie in Kapitel \ref{sec:vorteile_des_page_object_pattern} bereits angesprochen haben 
Leotta et al. \cite{leotta_repairing_2013} allerdings auch gezeigt, dass sich diese Investition, trotz der initial höheren Kosten, durchaus lohnen kann.

\newpage
\chapter{Teilautomatisierte Generierung von Page Objects}
\label{sec:teilautomatisierte_generierung_von_pageObjects}

Ein großer Teil des in Kapitel \ref{sec:probleme_des_page_object_pattern} angesprochenen initialen Mehraufwands bei der Verwendung des Page Object Pattern beläuft sich auf die Erstellung der Page Objects.
Wie in Listing \ref{lst:poCreatePage} und \ref{lst:poShowPage} zu sehen ist, handelt es sich bei Page Objects allerdings um nicht gerade komplexe Klassen. In der Praxis zeigt sich, dass ein Großteil der Arbeit darin besteht die verschiedenen Locatoren der Elemente aus dem Quelltext der Seite herauszufinden und in die Generische Form eines Page Objects zu überführen.
Diese Arbeit kostet zwar viel Zeit, ist allerdings nicht gerade anspruchsvoll.
Möchte man den initialen Mehraufwand bei der Verwendung des Page Object Pattern entgegenwirken bieten die Page Objects somit einen guten Ansatzpunkt.
Ihre Erstellung ist zeitaufwendig und weitestgehend generisch. Gute Voraussetzungen also um das Erzeugen der Pagae Objects zu Automatisieren.
\section{Übersicht über die Idee}

Selenium in Verbindung mit dem Page Object Pattern ist auch ein Teil des Technologiestacks des IT-Dienstleisters der Landeshauptstadt München (it@M) und wird dort zum Testen komplexer Webanwendungen verwendet. Auch bei it@M hat man in Bezug auf die Erstellung der Page Objects die Erfahrungen gemacht, dass es sich um eine generische und zeitaufwendige Arbeit handelt.
In Zusammenarbeit mit it@M wurde daher die Idee entwickelt, das Erstellen von Page Objects mit Hilfe einer Softwarelösung zu unterstützen. 
Anhand des Seitenquelltextes der zu testenden Webanwendung sollen die verschiedenen Elemente des Page Objects identifiziert und zur Generierung der Page Objects verwendet werden.
Zwei Ansätze wurden dabei diskutiert. Eine vollautomatisierte Generierung von Page Objects und eine teilautomatisierte Generierung.
Ein vollautomatischer Ansatz würde beinhalten, dass für einen übergebenen Seitenquelltext ohne weiteres Zutun ein vollständiges Page Object generiert wird. Dieser Ansatz hat jedoch mit zahlreichen Problemen zu kämpfen. Oft wird nur ein Bruchteil der Elemente einer Webseite für die Testfälle benötigt. Selenium kann aber prinzipiell jedes Element, dass im Seitenquelltext bereitgestellt wird, ansprechen. Bei einer vollautomatischen Generierung müssten daher entweder alle Elemente einer Seite übernommen oder eine definierte Auswahl getroffen werden.
Wird eine Auswahl getroffen besteht das Risiko, dass Elemente ausgelassen werden die vom Tester möglicherweise benötigt werden. Werden alle Elemente übernommen, werden die Page Objects schnell überladen und unübersichtlich. Das Überladen der Page Objects geschieht dann auf Kosten der Robustheit. Strukturelle Änderungen in der Website wirken sich auch auf die Locatoren der Elemente in den Page Objects aus. Um die Page Objects stabil zu halten, müssen diese bei Änderungen in der Seitenstruktur berichtigt werden.
Es ist daher nicht sinnvoll Elemente in den Page Objects zu pflegen, die nicht verwendet werden. Unbenutzte Elemente bedeuten entweder zusätzlichen Wartungsaufwand oder veralten unbemerkt.
Ein weiteres Problem des vollautomatischen Ansatzes stellen die Übergänge zwischen den Seiten einer Webanwendung dar. Interaktionen mit der Webanwendung wie beispielsweise das betätigen eines Button führen oft zum aufrufen einer neuen Seite. Im weiteren werden diese Übergänge als Transitions bezeichnet. Diese Transitions werden optimaler weise auch in den Page Objects abgebildet. Das Page Object gibt dazu das entsprechende Page Object der Zielseite als Rückgabe eines Methodenaufrufs zurück, wie es beispielsweise in der Methode createEntry() im Listing \ref{lst:poCreatePage} gezeigt ist. Allein aus dem Seitenquelltext zu ermitteln welche Seite das Ziel einer Transition ist erweist sich jedoch oft als sehr schwierig bis unmöglich.
Um die Komplexität des Projektes auf Grund der genannten Probleme nicht zu groß werden zu lassen wurde sich für einen teilautomatisierte Lösung entschieden.
Ziel ist es also nicht, automatisch ein vollständiges Page Object zu generieren sondern den Entwickler bei der Generierung der Page Objects zu unterstützen. Anhand des Quelltextes sollen dem Entwickler die möglichen Elemente der Seite in einer Vorauswahl bereitgestellt werden. Aus diesen Elementen können dann diejenigen ausgewählt werden, die vom Entwickler im späteren Page Object benötigt werden. Auf diese Weise wird eine Überladung der Page Objects verhindert und gleichzeitig sichergestellt, dass die Elemente vorhanden sind, die benötigt werden.
Ob es sich bei einem Element um eine Transition handelt, also ein Element welches auf eine neue Seite führt, muss auch nicht mehr automatisch anhand des Quelltextes ermittelt werden sonder kann vom Entwickler direkt bei der Auswahl der benötigten Elemente mit angegeben werden.
Die so vom Entwickler ausgewählten Informationen können dann verwendet werden um daraus das fertige Page Object zu generieren.
Im Rahmen des Projektes SeleniPo soll dieser Ansatz in Zusammenarbeit mit it@M in Form einer Denktopanwendung umgesetzt werden. 

\section{Abgrenzung zu bestehenden Ansätzen}
Sowohl für die vollautomatische Generierung von Page Objects als auch für eine teilautomatisierten Generierung gibt es bereits mögliche Lösungsansätze. 
Stocco et al. \cite{stocco_why_2015} beschreiben in einem Paper das von ihnen entwickeltes Framework APOGEN mit deren Hilfe Page Objects vollautomatisch generiert werden können. Die Generierung der Page Objects soll dabei weit über das Anlegen von Elementen hinausgehen und auch die Funktionalitäten der einzelnen Webseiten in Form von Methoden mit einschließen.
Das Framework analysiert dazu die Struktur der Webanwendung mittels eines Crawlers. Die Informationen die über den Crawler zusammengetragen wurden, wie beispielsweise das DOM der einzelnen Webseiten, werden anschließend über eine statische Analyse aufbereitet und für die Generierung der Page Objects verwendet.
Nach eigenen Angaben sollen ca. 75\% des von APOGEN generierten Codes ohne Anpassungen verwendet werden können. Die restlichen 25\% benötigen nur kleine Änderungen.
Bei APOGEN handelt es sich jedoch um ein noch sehr junges Projekt. Das Paper zu diesem Projekt wurde im Mai 2015 veröffentlicht. APOGEN ist daher eher einen Prototyp der zwar die Möglichkeiten aufzeigt die in diesem Bereich gegeben sind jedoch noch nicht für den produktiven Einsatz in einem großen unternehmen geeignet ist.
Nach eigenen Angaben Leidet das Projekt noch unter einigen Einschränkungen. Eine der genannten Einschränkungen ist die Limitierung durch den Crawler.
APOGEN kann nur Webseiten in Page Objects umwandeln, die auch durch den Crawler erreicht wurden.
Für einfache Webanwendungen stellt das kein großes Problem da, für sehr komplexe Anwendungen mit einer ausgeprägten logischen Validierung allerdings schon.
Viele Seiten die hinter logisch validierten Eingaben liegen können vom Crawler nicht erreicht werden und stehen somit für die Generierung nicht zur Verfügung.

Neben der vollautomatischen Generierung existieren noch eine Reihe von Open-Source-Framworks 
die einen teilautomatisierten Ansatz verfolgen, ähnlich wie es das Projekt SelneiPo erreichen will.
Stocco et al. \cite{stocco_why_2015} nennen in ihrem Paper die drei derzeit wichtigsten Vertreter:

\begin{itemize}
\item \textit{OHMAP} \cite{virtuetech_gmbh_ohmap_2015}: Bei OHMAP handelt es sich um eine online Webseite die es dem Benutzer erlaubt HTML-Code in eine Textarea zu Kopieren. Aus dem übergebenen HTML-Code generiert das Tool eine einfache Java-Klasse die für jedes gefundene Input-Feld ein WebElement enthält. Die Variablennamen werden dabei aus den HTML-Attributen gebildet. Als Locator wird ein einfacher XPath-Ausdruck verwendet.
	
\item \textit{SWD Page Recorder} \cite{dmytro_zharii_dzharii/swd-recorder_2015}: Der SWD Page Recorder ermöglicht es dem Benutzer eine beliebige Webanwendung zu starten und das GUI der Anwendung mit einem click\&record-Mechanismus zu inspizieren.
Nach jedem Klick auf das Interface der Anwendung wird ein drop-down-Menü angezeigt in welches manuell ein Name für das ausgewählte Element angegeben werden kann. Als Locator wird ein einfacher XPath-Ausdruck generiert.
Das so erstellte Modell der Anwendung kann in verschiedene Sprachen exportiert werden, wie beispielsweise Java, C\#, Rython, Ruby oder Perl. Beim SWD Page Recorder handelt es sich um eine .NET Anwendung.

\item\textit{ WTF PageObject Utility Chrome Extension} \cite{daniel_wiredrive/wtframework_2015}: WTF unterstützt den Entwickler beim erstellen von Page Objects indem Locatoren der Form id, name, CSS oder XPath erstellt werden. Der generierte Code ist in Python.
	
\end{itemize}

Der Technologiestack von it@M sieht eine Entwicklung der Selenium-Testfälle in Java vor. Als Betriebssystem kommt darüber hinaus Linux zum Einsatz.
Zwei der genannten Lösungsansetzen scheiden daher von vornherein für den produktiven Einsatz beim externen IT-Dienstleister der Landeshauptstadt München aus. Beim SWD Page Recorder handelt es sich um eine .NET Anwendung die nur schwer unter Linux betrieben werden kann. Die WTF PageObject Utility Chrome Extension kann nur im Python-Umfeld betrieben werden. OHMAP wäre aus technischer Sicht eine mögliche Lösungsalternative. Allerdings sind Komfort und Umfang der Anwendung aus sicht von it@M nicht ausreichend. Ohne eigne Konfiguration ist es mit OHMAP nur möglich input-Felder zu extrahieren.
Der HTML-Quelltext muss händisch aus der zu testenden Anwendung extrahiert werden. \\
OWAP als auch der SWD Page Recorder haben zusätzlich das Problem, dass die erzeugten XPath ausdrücke oft sehr einfach gewählt werden und damit sehr stark von der Position der Element innerhalb der Seite abhängig sind. Die eigentlichen Charakteristika der Elemente werden somit oft nicht beachtet. Listing \ref{lst:badXPath}
zeigt einen solche von OHMAP generierten XPath.

\begin{lstlisting}[caption={Exportierte Testfälle},label={lst:badXPath}]
	public class YourPageObjectName {
		//...		
 		@FindBy(xpath = "/html/body/div/div[1]/div[1]/h1/a[2]")
		public WebElement followVirtuetechGmbH;	
		//...
	}
\end{lstlisting}

Der zu adressierende Link in  Listing \ref{lst:badXPath} wird alleine über seine Position innerhalb des DOM der Seite bestimmt.
Um den Locator zu zerstören würde es alleine genügen ein weiteres div-Tag vor dem Link einzufügen.
\\
Ein weiteres Problem, dass sich die gezeigten Lösungen teilen ist, dass sie immer nur ein Page Object auf einmal betrachten. Transitionen, also Übergänge zwischen den einzelnen Webseiten der zu testenden Anwendung, werden somit nicht beachtet. Die dynamische Komponente der Anwendung wird beim generieren der Page Objects somit außer acht gelassen und muss nachträglich händisch hinzugefügt werden.

Mit SeleniPo soll der Versuch unternommen werden die Schwachstellen der bereits existierenden Lösungsansätzen zu verbessern und eine Plattformunabhängige Lösung zu schaffen, die in der IT-Infrastruktur von it@M betrieben werden kann.

\newpage
\section{SeleniPo - PageObject Generator}

Abbildung \ref{fig:poGenerator} zeigt die Denktopanwendung (PageObject Generator) die im Rahmen des Projektes SeleniPo entwickelt wurde. Mit Hilfe dieser Anwendung ist es möglich Page Objects teilautomatisiert zu generieren. Die Anwendung bietet dazu die Möglichkeit einen Browser zu starten und über vorgefertigte Selektoren die Webanwendung nach benötigten Elementen bzw. Transitionen zu durchsuchen. Auf diese Weise kann ein Modell der Anwendung erstellt werden, dass zur Generierung der Page Objects verwendet wird.

\begin{figure}[htb]
  \centering  
  \includegraphics[scale=0.5]{img/poGenerator.JPG}\\
  \caption{SeleniPo - PageObject Generator}
  \label{fig:poGenerator}
\end{figure}



Das Interface des PageObject Generators teilt sich in drei Bereiche:

\begin{itemize}
	\item Das aktuelle Page Object Modell (Abbildung \ref{fig:poGeneratorPo})
	\item Den HTML-Parser (Abbildung \ref{fig:poGeneratorHtml})
	\item Das Menü (Abbildung \ref{fig:poGeneratorMenu})
\end{itemize}


Abbildung \ref{fig:poGeneratorPo} zeigt den Bereich des Generators mit dem das aktuelle Page Object Modell der zu testenden Anwendung verwaltet werden kann. Mit diesem Bereich der Anwendung können neue Page Objects angelegt und bearbeitet werden. Elemente und Transitionen können manuell hinzugefügt, editiert oder gelöscht werden. Darüber hinaus bietet Die Anwendung die Möglichkeit existierende Elemente und Transitionen auf ihre Richtigkeit zu testen.

\begin{figure}[htb]
  \centering  
  \includegraphics[scale=0.5]{img/poGeneratorPo.JPG}\\
  \caption{SeleniPo - PageObject Generator - Page Object Model}
  \label{fig:poGeneratorPo}
\end{figure}

\newpage

Abbildung \ref{fig:poGeneratorHtml} zeigt den Bereich des Generators mit dem der Entwickler bei der Erstellung von Elementen und Transitionen im Page Object unterstützt werden kann.
Über den Button Start kann ein Browser gestartet werden. Über das Locator-Dropdown kann dann über vorgefertigte Selektoren die aktuell im Browser dargestellte Webseite nach Elementen bzw. Transitionen durchsucht werden. Im Page Object benötigte Elemente und Transitionen können dann in das ausgewählte Page Object im Page Object Modell übernommen werden.

\begin{figure}[htb]
  \centering  
  \includegraphics[scale=0.5]{img/poGeneratorHtml.JPG}\\
  \caption{SeleniPo - PageObject Generator - Html Parser}
  \label{fig:poGeneratorHtml}
\end{figure}

\newpage

Abbildung \ref{fig:poGeneratorMenu} markiert das Menü des PageObject Generator. Mit Hilfe des Menüs können Zwischenstände des Page Object Modells gespeichert und geladen werden.
Über das Menü kann darüber hinaus die Generierung der Page Objects aus dem aktuell geladenen Modell gestartet werden.

\begin{figure}[htb]
  \centering  
  \includegraphics[scale=0.5]{img/poGeneratorMenu.JPG}\\
  \caption{SeleniPo - PageObject Generator - Menü}
  \label{fig:poGeneratorMenu}
\end{figure}

\newpage




Screenshot..

Einordnung in Deploymentsicht
--> Beschreibe anhand von diagramm

Use-Cases: Beschreibe Standartusecase mit Sequenzdiagramm. (Pageobjekte anlegen -> kann um Elemente und Transitiionen Erweitert werden. Diese können auch Teilautomatisiert aus HTML generiert werden.

Use-Cases Diagramm 
Verbindung von Use-Case-Diagramm mit GUI

--> die einzelnen Usecases fachlich beschreiben.

Technische Sicht. 
Komponentendiagramm Allgemein erklären.

Jede Komponente abarbeiten in sinnvoller tiefe (z.B. Klassendiagram / Zustandsautomat ect.)





\section{Was gibt es bereits für Lösungen}
Die Stabilität einer Testsuite steht und fällt mit der Stabilität der locatoren. Daher ist der Ansatz der gezeigten Lösungen nicht immer der beste.


\section{einordnung des Testharness und gui in die Geamtstruktur (Deploymentdiagramm)}

\section{übersicht über Aufbau des Systems}

\subsection{pro modul ein kapitel}

\section{Vorteile und Probleme}

\section{Anwendung}

\newpage
\chapter{Fazit}
\label{sec:fazit}

\section{Ausblick}
\label{ausblick}
Ein Maximum an Zeitersparnis lässt sich mit dem Page Object Generator erreichen, wenn der Konfigurationsaufwand von Testprojekt und Generator möglichst gering gehalten wird.
Ein sinnvoller Ansatzpunkt um das Projekt SeleniPo weiter voranzutreiben wäre es daher die Entwicklung des Testharness zu Verbessern. Bei der Implementierung des Testharness wie er in Kapitel \ref{sec:selenipotestharness} beschrieben ist, handelt es sich um einen Prototypen der zwar als Grundlage für den produktiven Einsatz verwendet werden kann jedoch noch viel Verbesserungspotential bietet. Mit einem auf den Page Object Generator perfekt abgestimmten Testharness lässt sich der Konfigurationsaufwand des Generators auf nahezu null reduzieren. Das senkt die Hämschwelle für die Benutzung des Tools und steigert die Akzeptanz beim Benutzer.
Darüber hinaus lässt sich mit einem gut vorbereiteten Testharness die Verwendung von best practice Ansätzäen weiter unterstützt.
Ein weitere Richtung in die der Page Object Generator verbessert werden könnte, wäre es verschiedenen Templates für die Generierung der Page Objects anzubieten.
Die derzeit im Page Object Generator angebotenen Templates und damit die generierten Page Object Klassen haben sich vom Selenium vorgeschlagenen Standard für Page Objects entfernt. Selenium schlägt die Verwendung von Page Objects mit annotierten WebElements als variablen vor. Die mit Locatoren annotierten WebElements werden bei der Instanziierung über eine PageFactory-Klasse aufgelöst und befüllt.
Im Gegensatz dazu erzeugen die im Page Object Generator derzeit hinterlegten Templates PageObject-Klassen die an Stelle von WebElementes die Klasse Control verwenden und Abhängig von der Existent der Klasse ByFactory sind (siehe Kapitel \ref{sec:selenipotestharness}). Um die von Selenium angebotene PageFactory zu unterstützen müssen neue Templates geschaffen welche die von der PageFactory-Klasse benötigten Konventionen erfüllen.
Diese Templates hätten den Vorteil, dass der Page Object Generator auch außerhalb der Landeshauptstadt München, für die der Generator in erster Linie entwickelt wurde, leichter Verbreitung finden könnte. Die von den Templates aktuell streng vorgegebene Struktur für das Testprojekt ist bei erster Betrachtung nicht sofort verständlich da sie sich vom Standard entfernt hat und erschwert damit Benutzern welche an die Arbeit mit der PageFactory-Klasse gewohnt sind den Einstieg.














\newpage
\appendix
\chapter{Anhang}
\section{Anwendungsfallbeschreibung}
\label{anhang:anwendungsfallbeschreibung}
\subsection{Neues Page Object anlegen}
\label{sec:neues_page_object_anlegen}

\begin{tabular}[h]{|p{4cm}|p{11,5cm}|}
\hline 
\rule[-1ex]{0pt}{2.5ex}Kurzbeschreibung: & 
Entwickler legt ein neues Page Object an. \\  
\hline 
\rule[-1ex]{0pt}{2.5ex}Akteure: & 
Entwickler \\ 
\hline 
\rule[-1ex]{0pt}{2.5ex}Motivation: & 
Entwickler benötigt Page Object in den Testfällen. \\ 
\hline 
\rule[-1ex]{0pt}{2.5ex}Vorbedingung: &  \\ 
\hline 
\rule[-1ex]{0pt}{2.5ex}Eingehende Daten: & Name und Paket des Page Object. \\ 
\hline 
\rule[-1ex]{0pt}{2.5ex}Ergebnisse: & Page Object ist ausgewählt. \\ 
\hline 
\rule[-1ex]{0pt}{2.5ex}Nachbedingungen: & Page Object wurde im Modell der Anwendung angelegt.  \\ 
\hline 
\end{tabular} 

\paragraph{Ablauf}

\begin{itemize}[itemsep=0pt]
\item[1.] Entwickler startet den Vorgang zum Anlegen eines neuen Page Object. 
\item[2.] System zeigt Dialog an. 
\item[3.] Entwickler trägt Namen des Page Object im Dialog ein.
\item[4.] Entwickler bestätigt den Dialog.
\item[5.] System prüft Eingaben.
\item[6.] System wählt Page Object aus.
\end{itemize}

\paragraph{Vorgang abgebrochen}
Statt Schritt 4-6:
\begin{itemize}[itemsep=0pt]
\item[4.] Entwickler bricht Vorgang ab. 
\item[5.] System ändert internen Zustand nicht. 
\end{itemize}

\paragraph{Validierung fehlgeschlagen}
Statt Schritt 6:
\begin{itemize}[itemsep=0pt]
\item[6.] System zeigt Fehlermeldung an. 
\item[7.] Entwickler bestätigt Fehlermeldung. 
\end{itemize}
Weiter mit Punkt 3. 

\paragraph{Paketstruktur des Page Object verfeinern}
Statt Schritt 4:
\begin{itemize}[itemsep=0pt]
\item[4.] Entwickler trägt Paket des Page Object ein.
\end{itemize}
Weiter mit Punkt 4. 
 

\subsection{Neues Element hinzufügen}
\label{sec:neues_element_hinzufügen}

\begin{tabular}[h]{|p{4cm}|p{11,5cm}|}
\hline 
\rule[-1ex]{0pt}{2.5ex}Kurzbeschreibung: & 
Entwickler legt ein Element in einem Page Object an. \\  
\hline 
\rule[-1ex]{0pt}{2.5ex}Akteure: & 
Entwickler \\ 
\hline 
\rule[-1ex]{0pt}{2.5ex}Motivation: & 
Entwickler benötigt Element der Seite in den Testfällen. \\ 
\hline 
\rule[-1ex]{0pt}{2.5ex}Vorbedingung: & 
Page Object bereits angelegt.\\ 
\hline 
\rule[-1ex]{0pt}{2.5ex}Eingehende Daten: & Interner Name des Elements, Art des Selektors, Locator, der das Element in der Seite identifiziert \\ 
\hline 
\rule[-1ex]{0pt}{2.5ex}Ergebnisse: & Element wird in den Elementen des Page Object angezeigt. \\ 
\hline 
\rule[-1ex]{0pt}{2.5ex}Nachbedingungen: & Element wurde im Modell der Anwendung angelegt.  \\ 
\hline 
\end{tabular} 

\paragraph{Ablauf}

\begin{itemize}[itemsep=0pt]
\item[1.] Entwickler wählt Page Object aus.
\item[2.] Entwickler wählt Bereich für die Elemente des Page Objects aus. 
\item[3.] System wechselt in den internen Zustand zum Bearbeiten von Elementen.
\item[4.] Entwickler startet den Vorgang zum Anlegen eines neuen Eintrags.
\item[5.] System zeigt Dialog an. 
\item[6.] Entwickler befüllt Dialog.
\item[7.] Entwickler bestätigt den Dialog.
\item[8.] System prüft, ob alle Felder befüllt sind.
\item[9.] System zeigt Element in den Elementen des Page Object an.
\end{itemize}

\paragraph{Vorgang abgebrochen}
Statt Schritt 7-9:
\begin{itemize}[itemsep=0pt]
\item[7.] Entwickler bricht Vorgang ab. 
\item[8.] System ändert internen Zustand nicht. 
\end{itemize}

\paragraph{Validierung fehlgeschlagen}
Statt Schritt 9:
\begin{itemize}
\item[9.] System zeigt Fehlermeldung an. 
\item[10.] Entwickler bestätigt Fehlermeldung. 
\end{itemize}
Weiter mit Punkt 6. 


\subsection{Neue Transition hinzufügen}
\label{sec:neue_transition_hinzufügen}

\begin{tabular}[h]{|p{4cm}|p{11,5cm}|}
\hline 
\rule[-1ex]{0pt}{2.5ex}Kurzbeschreibung: & 
Entwickler legt einen Seitenübergang zu einem anderen Page Object an. \\  
\hline 
\rule[-1ex]{0pt}{2.5ex}Akteure: & 
Entwickler \\ 
\hline 
\rule[-1ex]{0pt}{2.5ex}Motivation: & 
Entwickler benötigt einen Übergang zu einer anderen Seite in den Testfällen. \\ 
\hline 
\rule[-1ex]{0pt}{2.5ex}Vorbedingung: & 
Page Object bereits angelegt. Page Object, das Ziel des Seitenübergangs ist, wurde bereits angelget.\\ 
\hline 
\rule[-1ex]{0pt}{2.5ex}Eingehende Daten: & Interner Name des Elements, Art des Selektors, Locator der das Element in der Seite identifiziert, Ziel Page Object. \\ 
\hline 
\rule[-1ex]{0pt}{2.5ex}Ergebnisse: & Transition wird in den Transitionen des Page Object angezeigt. \\ 
\hline 
\rule[-1ex]{0pt}{2.5ex}Nachbedingungen: & Transition wurde im Modell der Anwendung angelegt.  \\ 
\hline 
\end{tabular} 

\paragraph{Ablauf}

\begin{itemize}[itemsep=0pt]
\item[1.] Entwickler wählt Page Object aus.
\item[2.] Entwickler wählt Bereich für die Transitionen des Page Objects aus. 
\item[3.] System wechselt in den internen Zustand zum Bearbeiten von Transitionen.
\item[4.] Entwickler startet den Vorgang zum Anlegen eines neuen Eintrags.
\item[5.] System zeigt Dialog an. 
\item[6.] Entwickler befüllt Dialog.
\item[7.] Entwickler bestätigt den Dialog.
\item[8.] System prüft ob alle Felder befüllt sind.
\item[9.] System zeigt Transition in den Transitionen des Page Object an.
\end{itemize}

\paragraph{Vorgang abgebrochen}
Statt Schritt 7-9:
\begin{itemize}[itemsep=0pt]
\item[7.] Entwickler bricht Vorgang ab. 
\item[8.] System ändert internen Zustand nicht. 
\end{itemize}

\paragraph{Validierung fehlgeschlagen}
Statt Schritt 9:
\begin{itemize}
\item[9.] System zeigt Fehlermeldung an. 
\item[10.] Entwickler bestätigt Fehlermeldung. 
\end{itemize}
Weiter mit Punkt 6. 


\subsection{Element aus HTML übernehmen}
\label{sec:element_from_html}

\begin{tabular}[h]{|p{4cm}|p{11,5cm}|}
\hline 
\rule[-1ex]{0pt}{2.5ex}Kurzbeschreibung: & 
Entwickler legt teilautomatisiert ein neues Element im Page Object an. \\  
\hline 
\rule[-1ex]{0pt}{2.5ex}Akteure: & 
Entwickler \\ 
\hline 
\rule[-1ex]{0pt}{2.5ex}Motivation: & 
Entwickler benötigt Element der Seite in den Testfällen. \\ 
\hline 
\rule[-1ex]{0pt}{2.5ex}Vorbedingung: & 
Page Object bereits angelegt. \\ 
\hline 
\rule[-1ex]{0pt}{2.5ex}Eingehende Daten: & Art des Locators, nach dem die Seite durchsucht werden soll. \\ 
\hline 
\rule[-1ex]{0pt}{2.5ex}Ergebnisse: & Element wird in den Elementen des Page Object angezeigt. \\ 
\hline 
\rule[-1ex]{0pt}{2.5ex}Nachbedingungen: & Element wurde im Modell der Anwendung angelegt.  \\ 
\hline 
\end{tabular} 

\paragraph{Ablauf}

\begin{itemize}[itemsep=0pt]
\item[1.] Entwickler wählt Page Object aus.
\item[2.] Entwickler startet aus der Anwendung heraus den Webbrowser mit der zum ausgewählten Page Object korrespondierenden Seite. 
\item[3.] Entwickler wählt Art des Locators, für den die Seite durchsucht werden soll, aus.
\item[4.] Entwickler wählt Aktion zum Analysieren der ausgewählten Webseite aus.
\item[5.] System zeigt die für den ausgewählten Locator auf der Seite identifizierten Elemente an.
\item[6.] Entwickler wählt gewünschtes Element aus den Treffern aus. 
\item[7.] Entwickler wählt Aktion zum Übernehmen des Elements in das Page Object aus.
\item[8.] System zeigt Element in den Elementen des Page Object an.
\end{itemize}

\paragraph{Testen des Elements}
Statt Schritt 7-8:
\begin{itemize}
\item[7.] Entwickler wählt Aktion zum Testen des ausgewählten Elements aus. 
\item[8.] System zeigt an, ob das Element auf der ausgewählten Webseite erfolgreich erreicht werden konnte. 
\end{itemize}
Weiter mit Punkt 7. 

\subsection{Transition aus HTML übernehmen}
\label{sec:Transition_from_html}

\begin{tabular}[h]{|p{4cm}|p{11,5cm}|}
\hline 
\rule[-1ex]{0pt}{2.5ex}Kurzbeschreibung: & 
Entwickler legt teilautomatisiert einen neuen Seitenübergang an. \\  
\hline 
\rule[-1ex]{0pt}{2.5ex}Akteure: & 
Entwickler \\ 
\hline 
\rule[-1ex]{0pt}{2.5ex}Motivation: & 
Entwickler benötigt einen Übergang zu einer anderen Seite in den Testfällen. \\ 
\hline 
\rule[-1ex]{0pt}{2.5ex}Vorbedingung: & 
Page Object bereits angelegt. Page Object, das Ziel des Seitenübergangs ist, wurde bereits angelget. \\ 
\hline 
\rule[-1ex]{0pt}{2.5ex}Eingehende Daten: & Art des Locators nach dem die Seite durchsucht werden soll, Ziel Page Object. \\ 
\hline 
\rule[-1ex]{0pt}{2.5ex}Ergebnisse: & Transition wird in den Transitionen des Page Object angezeigt. \\ 
\hline 
\rule[-1ex]{0pt}{2.5ex}Nachbedingungen: & Transition wurde im Modell der Anwendung angelegt.  \\ 
\hline 
\end{tabular} 

\paragraph{Ablauf}

\begin{itemize}[itemsep=0pt]
\item[1.] Entwickler wählt passendes Page Object aus.
\item[2.] Entwickler startet aus der Anwendung heraus den Webbrowser mit der zum ausgewählten Page Object korrespondierenden Seite. 
\item[3.] Entwickler wählt Art des Locators, für den die Seite durchsucht werden soll, aus.
\item[4.] Entwickler wählt Aktion zum Analysieren der ausgewählten Webseite aus.
\item[5.] System zeigt die für den ausgewählten Locator auf der Seite identifizierten Elemente an.
\item[6.] Entwickler wählt gewünschtes Element aus den Treffern aus. 
\item[7.] Entwickler wählt Aktion zum Übernehmen des Elements als Transition in das Page Object aus.
\item[8.] System zeigt Dialog mit den Informationen des ausgewählten Elements an.
\item[9.] Entwickler ergänzt die Informationen um das Ziel Page Object der Transition.
\item[10.] Entwickler bestätigt den Dialog.
\item[11.] System prüft, ob alle Felder befüllt sind.
\item[12.] System zeigt Element in den Transitionen des Page Object an.
\end{itemize}

\paragraph{Testen der Transition}
Statt Schritt 7-12:
\begin{itemize}
\item[7.] Entwickler wählt Aktion zum Testen des ausgewählten Elements aus. 
\item[8.] System zeigt an, ob das Element auf der ausgewählten Webseite erfolgreich erreicht werden konnte. 
\end{itemize}
Weiter mit Punkt 7.

\paragraph{Vorgang abgebrochen}
Statt Schritt 10-12:
\begin{itemize}[itemsep=0pt]
\item[10.] Entwickler bricht Vorgang ab. 
\item[11.] System ändert internen Zustand nicht. 
\end{itemize}

\paragraph{Validierung fehlgeschlagen}
Statt Schritt 12:
\begin{itemize}
\item[12.] System zeigt Fehlermeldung an. 
\item[13.] Entwickler bestätigt Fehlermeldung. 
\end{itemize}
Weiter mit Punkt 6. 


\subsection{Vorhandenen Eintrag editieren}
\label{sec:edit_entry}

\begin{tabular}[h]{|p{4cm}|p{11,5cm}|}
\hline 
\rule[-1ex]{0pt}{2.5ex}Kurzbeschreibung: & 
Entwickler editiert bereits vorhandenen Eintrag. \\  
\hline 
\rule[-1ex]{0pt}{2.5ex}Akteure: & 
Entwickler \\ 
\hline 
\rule[-1ex]{0pt}{2.5ex}Motivation: & 
Entwickler möchte einen bereits vorhandenen Eintrag überarbeiten. \\ 
\hline 
\rule[-1ex]{0pt}{2.5ex}Vorbedingung: & 
Eintrag bereits vorhanden. \\ 
\hline 
\rule[-1ex]{0pt}{2.5ex}Eingehende Daten: & Geänderte Werte. \\ 
\hline 
\rule[-1ex]{0pt}{2.5ex}Ergebnisse: & Eintrag wird in aktualisierter Form angezeigt. \\ 
\hline 
\rule[-1ex]{0pt}{2.5ex}Nachbedingungen: & Eintrag wurde im Modell der Anwendung aktualisiert.  \\ 
\hline 
\end{tabular} 

\paragraph{Ablauf}

\begin{itemize}[itemsep=0pt]
\item[1.] Entwickler wählt den zu editierenden Eintrag aus.
\item[2.] Entwickler löst die Aktion zum Editieren der aktuellen Auswahl aus. 
\item[3.] System zeigt Dialog mit den aktuellen Werten des Eintrags an.
\item[4.] Entwickler nimmt die gewünschten Änderungen vor.
\item[5.] Entwickler bestätigt den Dialog.
\item[6.] System prüft die Felder des Dialogs.
\item[7.] System zeigt den Eintrag in aktualisierter Form an.
\end{itemize}

\paragraph{Vorgang abgebrochen}
Statt Schritt 5-7:
\begin{itemize}[itemsep=0pt]
\item[5.] Entwickler bricht Vorgang ab. 
\item[6.] System ändert internen Zustand nicht. 
\end{itemize}

\paragraph{Validierung fehlgeschlagen}
Statt Schritt 7:
\begin{itemize}[itemsep=0pt]
\item[7.] System zeigt Fehlermeldung an. 
\item[8.] Entwickler bestätigt Fehlermeldung. 
\end{itemize}
Weiter mit Punkt 4. 

\subsection{Vorhandenen Eintrag löschen}
\label{sec:delete_entry}

\begin{tabular}[h]{|p{4cm}|p{11,5cm}|}
\hline 
\rule[-1ex]{0pt}{2.5ex}Kurzbeschreibung: & 
Entwickler löscht bereits vorhandenen Eintrag. \\  
\hline 
\rule[-1ex]{0pt}{2.5ex}Akteure: & 
Entwickler \\ 
\hline 
\rule[-1ex]{0pt}{2.5ex}Motivation: & 
Entwickler möchte einen bereits vorhandenen Eintrag löschen. \\ 
\hline 
\rule[-1ex]{0pt}{2.5ex}Vorbedingung: & 
Eintrag bereits vorhanden. \\ 
\hline 
\rule[-1ex]{0pt}{2.5ex}Eingehende Daten: & \\ 
\hline 
\rule[-1ex]{0pt}{2.5ex}Ergebnisse: & Eintrag wird nicht mehr angezeigt. \\ 
\hline 
\rule[-1ex]{0pt}{2.5ex}Nachbedingungen: & Eintrag wurde aus dem Modell der Anwendung entfernt.  \\ 
\hline 
\end{tabular} 

\paragraph{Ablauf}

\begin{itemize}[itemsep=0pt]
\item[1.] Entwickler wählt den zu löschenden Eintrag aus.
\item[2.] Entwickler löst die Aktion zum Löschen der aktuellen Auswahl aus. 
\item[3.] System entfernt den ausgewählten Eintrag.
\end{itemize}

\paragraph{Zu löschender Eintrag ist gesamtes Page Object}
Statt Schritt 3:
\begin{itemize}[itemsep=0pt]
\item[3.] System zeigt einen Bestätigungsdialog an. 
\item[4.] Entwickler bestätigt Dialog. 
\end{itemize}
Weiter mit Punkt 3. 

\paragraph{Vorgang abgebrochen}
Statt Schritt 4 des Sonderfalls \grq Zu löschender Eintrag ist gesamtes Page Object\grq:
\begin{itemize}[itemsep=0pt]
\item[4.] Entwickler bricht Vorgang ab. 
\item[5.] System ändert internen Zustand nicht. 
\end{itemize}

\subsection{Vorhandenen Eintrag testen}
\label{sec:test_entry}

\begin{tabular}[h]{|p{4cm}|p{11,5cm}|}
\hline 
\rule[-1ex]{0pt}{2.5ex}Kurzbeschreibung: & 
Entwickler testet einen vorhandenen Eintrag. \\  
\hline 
\rule[-1ex]{0pt}{2.5ex}Akteure: & 
Entwickler \\ 
\hline 
\rule[-1ex]{0pt}{2.5ex}Motivation: & 
Entwickler möchte die Erreichbarkeit eines bereits vorhandenen Eintrags prüfen. \\ 
\hline 
\rule[-1ex]{0pt}{2.5ex}Vorbedingung: & 
Eintrag bereits vorhanden. \\ 
\hline 
\rule[-1ex]{0pt}{2.5ex}Eingehende Daten: & \\ 
\hline 
\rule[-1ex]{0pt}{2.5ex}Ergebnisse: & Eintrag ist mit dem Testergebnis markiert. \\ 
\hline 
\rule[-1ex]{0pt}{2.5ex}Nachbedingungen: & Modell der Anwendung ist unverändert.  \\ 
\hline 
\end{tabular} 

\paragraph{Ablauf}

\begin{itemize}[itemsep=0pt]
\item[1.] Entwickler startet aus der Anwendung heraus den Webbrowser auf der dem Page Object Korrespondierenden Webseite. 
\item[2.] Entwickler wählt den zu testenden Eintrag aus.
\item[3.] Entwickler löst die Aktion zum Testen der aktuellen Auswahl aus. 
\item[4.] System zeigt Ergebnis des Tests an.
\end{itemize}


\subsection{Laden eines vorhandenen Modells}
\label{sec:load}

\begin{tabular}[h]{|p{4cm}|p{11,5cm}|}
\hline 
\rule[-1ex]{0pt}{2.5ex}Kurzbeschreibung: & 
Entwickler lädt ein zuvor angelegtes und gespeichertes Modell. \\  
\hline 
\rule[-1ex]{0pt}{2.5ex}Akteure: & 
Entwickler \\ 
\hline 
\rule[-1ex]{0pt}{2.5ex}Motivation: & 
Entwickler möchte einen alten Speicherstand laden. \\ 
\hline 
\rule[-1ex]{0pt}{2.5ex}Vorbedingung: & 
Save Datei vorhanden. \\ 
\hline 
\rule[-1ex]{0pt}{2.5ex}Eingehende Daten: & Save-File\\ 
\hline 
\rule[-1ex]{0pt}{2.5ex}Ergebnisse: & Zustand der Save-File wiederhergestellt. \\ 
\hline 
\rule[-1ex]{0pt}{2.5ex}Nachbedingungen: & Modell der Anwendung wurde mit den Werten aus der Save-File befüllt.  \\ 
\hline 
\end{tabular} 

\paragraph{Ablauf}

\begin{itemize}[itemsep=0pt]
\item[1.] Entwickler wählt die Aktion zum Laden einer Save-File aus.
\item[2.] System öffnet Auswahldialog. 
\item[3.] Entwickler wählt Save-File aus.
\item[4.] System zeigt die Einträge der Save-File an.
\end{itemize}

\paragraph{Vorgang abgebrochen}
Statt Schritt 3:
\begin{itemize}[itemsep=0pt]
\item[3.] Entwickler bricht Vorgang ab. 
\item[4.] System ändert internen Zustand nicht. 
\end{itemize}

\subsection{Speichern eines vorhandenen Modells}
\label{sec:save}

\begin{tabular}[h]{|p{4cm}|p{11,5cm}|}
\hline 
\rule[-1ex]{0pt}{2.5ex}Kurzbeschreibung: & 
Entwickler speichert einen Stand der Anwendung. \\  
\hline 
\rule[-1ex]{0pt}{2.5ex}Akteure: & 
Entwickler \\ 
\hline 
\rule[-1ex]{0pt}{2.5ex}Motivation: & 
Entwickler möchte einen Stand der Anwendung zur späteren Wiederherstellbarkeit speichern. \\ 
\hline 
\rule[-1ex]{0pt}{2.5ex}Vorbedingung: &  \\ 
\hline 
\rule[-1ex]{0pt}{2.5ex}Eingehende Daten: & \\ 
\hline 
\rule[-1ex]{0pt}{2.5ex}Ergebnisse: & Save-File wurde angelegt. \\ 
\hline 
\rule[-1ex]{0pt}{2.5ex}Nachbedingungen: & Save-File im Zielpfad vorhanden.  \\ 
\hline 
\end{tabular} 

\paragraph{Ablauf}

\begin{itemize}[itemsep=0pt]
\item[1.] Entwickler wählt die Aktion zum Speichern eines Zwischenstands aus.
\item[2.] System öffnet Auswahldialog. 
\item[3.] Entwickler wählt Ort zum Ablegen der Save-File aus.
\item[4.] Entwickler gibt Namen für die Save-File an.
\item[5.] Entwickler bestätigt Dialog.
\item[6.] System erzeugt Save-File.
\end{itemize}

\paragraph{Vorgang abgebrochen}
Statt Schritt 5-6:
\begin{itemize}[itemsep=0pt]
\item[5.] Entwickler bricht Vorgang ab. 
\item[6.] System ändert internen Zustand nicht. 
\end{itemize}

\subsection{Page Objects generieren}
\label{sec:generate_page_objects}

\begin{tabular}[h]{|p{4cm}|p{11,5cm}|}
\hline 
\rule[-1ex]{0pt}{2.5ex}Kurzbeschreibung: & 
Entwickler erzeugt Quellcode aus dem in der Anwendung erzeugten Modell. \\  
\hline 
\rule[-1ex]{0pt}{2.5ex}Akteure: & 
Entwickler \\ 
\hline 
\rule[-1ex]{0pt}{2.5ex}Motivation: & 
Entwickler möchte aus einem Stand der Anwendung Page Object Klassen generieren. \\ 
\hline 
\rule[-1ex]{0pt}{2.5ex}Vorbedingung: & 
Page Objects vorhanden. \\
\hline 
\rule[-1ex]{0pt}{2.5ex}Eingehende Daten: & \\ 
\hline 
\rule[-1ex]{0pt}{2.5ex}Ergebnisse: & Page Object Klassen wurden erzeugt. \\ 
\hline 
\rule[-1ex]{0pt}{2.5ex}Nachbedingungen: & Page Objects im Zielpfad vorhanden.  \\ 
\hline 
\end{tabular} 

\paragraph{Ablauf}

\begin{itemize}[itemsep=0pt]
\item[1.] Entwickler wählt die Aktion zum Generieren der Page Objects aus.
\item[2.] System öffnet Auswahldialog. 
\item[3.] Entwickler wählt Ort, in dem die Page Objects abgelegt werden sollen.
\item[4.] System generiert aus den Informationen des aktuellen Modells der Anwendung die entsprechenden Page Object Klassen.

\end{itemize}

\paragraph{Vorgang abgebrochen}
Statt Schritt 3:
\begin{itemize}[itemsep=0pt]
\item[3.] Entwickler bricht Vorgang ab. 
\item[4.] System ändert internen Zustand nicht. 
\end{itemize}

\newpage
\section{Zustände des Page Object Generator}
\label{sec:zustände_des_page_object_generator}


\begin{figure}[htb]
  \centering  
  \includegraphics[scale=0.43]{img/StateMashine.jpg}\\
  \caption{Zustandsmodell des SeleniPoEditor}
  \label{fig:state_mashine}
\end{figure}

\begin{figure}[htb]
  \centering  
  \includegraphics[scale=0.4]{img/poGeneratorEvents.jpg}\\
  \caption{Zuordnung der in Abbildung \ref{fig:state_mashine} verwendeten Namen }
  \label{fig:state_mashine_zuordnung}
\end{figure}

\newpage
\section{Vollständiges technisches Modell}
\label{anhang:vollständiges_technisches_modell}

\includegraphics[angle=90,height=0.95\textheight]{img/ComplexModel.jpg}



\newpage

\section{Beispiel für ein Velocity Template}
\label{anhang:beispiel_velocity_template}
\begin{lstlisting}[caption={poGenerated.vm},label={lst:template_pogenerated}]
#set( $BASE_PACKAGE = $basePackagePath )
#set( $GENERATED_PACKAGE = "$BASE_PACKAGE#if ( $poGeneric.getPackageName() && $poGeneric.getPackageName().trim().size() != 0 ).$poGeneric.getPackageName()#end" )
#set( $EDIT_PACKAGE = "$BASE_PACKAGE" )
#set( $CLASS_NAME = "$poGeneric.getIdentifier()Generated" )

package $GENERATED_PACKAGE;

import ${BASE_PACKAGE}.BasePo;
import ${BASE_PACKAGE}.Control;
import ${BASE_PACKAGE}.PageObject;
#foreach( $destPo in $destionationPos )
import $EDIT_PACKAGE#if ( $destPo.getPackageName() && $destPo.getPackageName().trim().size() != 0 ).$destPo.getPackageName()#end.$destPo.getIdentifier();
#end

public class $CLASS_NAME extends BasePo {

#foreach( $element in $poGeneric.getElements() )
	public final Control $element.getIdentifier() = control(by.$element.getType().getByMethodName()("$element.getLocator()"));
#end
#foreach( $transition in $poGeneric.getTransitions() )
	public final Control $transition.getIdentifier() = control(by.$transition.getType().getByMethodName()("$transition.getLocator()"));
#end
	public $CLASS_NAME(PageObject po) {
		super(po);
	}
#foreach( $transition in $poGeneric.getTransitions() )
	
	public $transition.getDestination().getIdentifier() click${display.capitalize($transition.getIdentifier())}() {
		${transition.getIdentifier()}.click();
		return new $transition.getDestination().getIdentifier()(this);
	}
#end  
#foreach( $element in $poGeneric.getElements() )
	/**
	 * Get the Control for the Element $transition.getIdentifier().
	 * @return $transition.getIdentifier() - Element
	 */
	public Control get${display.capitalize($element.getIdentifier())}() {
		return $element.getIdentifier();
	}	
#end
#foreach( $transition in $poGeneric.getTransitions() )
	/**
	 * Get the Control for the Transition $transition.getIdentifier().
	 * @return $transition.getIdentifier() - Transition
	 */
	public Control get${display.capitalize($transition.getIdentifier())}() {
		return $transition.getIdentifier();
	}
	
#end
}

\end{lstlisting} 
\begin{lstlisting}[caption={poEditable.vm},label={lst:template_poeditable}]
#set( $BASE_PACKAGE = $basePackagePath )
#set( $EDIT_PACKAGE = "$BASE_PACKAGE#if ( $poGeneric.getPackageName() && $poGeneric.getPackageName().trim().size() != 0 ).$poGeneric.getPackageName()#end" )
#set( $GENERATED_PACKAGE = "$BASE_PACKAGE#if ( $poGeneric.getPackageName() && $poGeneric.getPackageName().trim().size() != 0 ).$poGeneric.getPackageName()#end" )
#set( $CLASS_NAME = "$poGeneric.getIdentifier()" )
#set( $CLASS_NAME_SUPER = "$poGeneric.getIdentifier()Generated" )

package $EDIT_PACKAGE;

import ${BASE_PACKAGE}.BasePo;
import ${BASE_PACKAGE}.Control;
import ${BASE_PACKAGE}.PageObject;
import $GENERATED_PACKAGE.$CLASS_NAME_SUPER;

public class $CLASS_NAME extends $CLASS_NAME_SUPER {

	public $CLASS_NAME(PageObject po) {
		super(po);
	}
}
\end{lstlisting} 


%Literaturverzeichnis
\newpage
\printbibliography

\end{document}
